Problem: Beschreibung von Permutationsgruppen, Aufzählung aller Elemente ist nur selben möglich ($S_{\uline{100}} = 100 \implies 10^{100} \dots 10^{200}$)\\
Aufzählung: Beschreibung als Automorphismusgruppen (siehe Kapitel 4 und 5) oder durch EZS.\\
Wiederholung:
\begin{definition}\label{sec3:def:generatingset}
	$U \subseteq G$ heißt \begriff{Erzeugendensystem} einer Gruppe $G :\equival$ jedes $g \in G$ ist als endliches Produkt $u_1, \dots, u_m$ mit $u_i \in U$ oder $u^{-1}_i \in U$ darstellbar.\\
	Bezeichunung: $G = \langle u_i \rangle_G$.
\end{definition}
Probleme:
\begin{itemize}
	\item[(P1)] Entscheide $g \in \gen{U}$ für $g \in S_n, U \subseteq S_n$?
	\item[(P2)] Bestimme Bahnen von $\gen{U}$, spezielle Bahnen $a^{\gen{U}}$ für spezielle $a \in G$.
	\item[(P3)] Beschreibung der Untergruppen von $\gen{U}$, benutze Methode von \begriff{SIMS} für große $G$. Man benutzt Menge $T_i$ für $i =\set{1,\dots,r}$, sodass
	\begin{align*}
		G = T_r \cdot T_{r-1} \cdots T_1
	\end{align*}
	und Darstellung
	\begin{align*}
		g = t_r \cdot t_{r-1} \cdot t_{r-2}\cdots t_1
	\end{align*}
	ist Eindeuting.
\end{itemize}
Damit wäre die Speicherformel: $\sum_{i=1}^r \abs{T_i}$
\begin{*example}
	$G = S_{\uline{n}}$ impliziert $\abs{G} = n!$ oder $\sum \abs{T_i} \le \frac{n(n+1)}{2}$ möglich, also ist der Speicherbedarf $\sim n^2$.
\end{*example}
\begin{definition}[\person{Sims}-Kette, \person{Sims}-Basis, Transversale]
	\label{sec3:def:sims_stuff}
	Die Sims-Kette einer Permutationsgruppe $G \subseteq S_M$, $M = \set{a_1, \dots, a_n}$ speziell $M = \uline{n} = \set{1, \dots, n}$\\
	für punktweise Stabilisatoren:
	\begin{align*}
		\begin{matrix}
			U_1 = G_{n_1} & U_2 = G_{n_1,n_2} & \dots & U_i = G_{n_1,\dots,n_i} &
			U_n = G_{n_1,\dots,n_n} = \set{e}
		\end{matrix}
	\end{align*}
	Also haben wir
	\begin{align*}
		\set{e} = U_1 \subseteq U_2 \subseteq \cdots \subseteq U_i \subseteq \cdots \subseteq U = G
	\end{align*}
	Sei $r:= \min\set{i \mid U_i = \set{e}}$ (hängt von der Reihenfolge der Elemente $n_i$ ab). Die Menge der $\set{a_1, \dots, a_r}$ genauer $(a_1,\dots, a_r)$ heißt \begriff{SIMS-Basis} von $G$ und
	\begin{align*}
		\set{e} = U_r \lneqq U_{r-1} \leqq \cdots \leqq U_1 \leqq U_0 = 0
	\end{align*}
	ist die \begriff{SIMS-Kette} von $G$ der Länge $r$ (zur Basis $(a_1, \dots, a_r)$). Für
	\begin{align*}
		U_{i-1}/U_i = U_i g_{i_1} \dcup U_i g_{i_2} \dcup U_i g \dcup \dots \dcup U_i g_{i_{n_i}}\quad \text{meist }g_{i_1} = e
	\end{align*}
	wird Repräsentatensystem (\begriff{Transversale}) $T_i := \set{g_{i_1}, \dots, g_{i_{n_i}}}$ gewählt $(i = 1, \dots, r)$.\\
	Beachte: 
	\begin{align*}
		U_{r-1}/U_r \cong U_{r-1}\text{,also } T_r = U_{r-1}
	\end{align*}
\end{definition}
Bei Umnummerierung der Elemente entstehen möglicherweise kürzere Base. (Fixpunkte in Basis weggelassen)
\begin{proposition}
		\label{sec3:prop:sims_chain_unique}
	Seien $G,T_i$ wie in \cref{sec3:def:sims_stuff}. Dann gilt
	\begin{defenum}
		\item Jede Permutation $g \in G$ lässt sich eindeutig in der Form
		\begin{align*}
			g = h_r h_{r-1} \cdots h_1 \mit h_i \in T_i\;(i \in \set{1, \dots, r})
		\end{align*}
		darstellen. Insbesondere gilt dann
		\begin{align*}
			G = T_r T_{r-1} \cdot \dots \cdot T_1 \nd \abs{G} = \prod_{i=1}^r n_i.
		\end{align*} \label{sec3:prop3:unique_simsform}
		\item Jede Permutation $g \in G$ ist eindeutig durch die Bilder der Basis festgelegt, d.h. durch $(a_1^g, \dots, a_r^g)$.\label{sec3:def:unique_image}
	\end{defenum}
\end{proposition}
\begin{*remark}
	\cref{sec3:prop3:unique_simsform} impliziert $T_1 \cup \dots \cup T_r$ ist ein (spezielles) Erzeugendensystem für $G$.
\end{*remark}
\begin{proof}
	\begin{itemize}
		\item zu \cref{sec3:prop3:unique_simsform}:
		\begin{align*}
			g \in G \implies \exists! h_1 \in T_1 \colon g \in U_1 h_1\\
			\implies gh^{-1}_1 \in U_1 \implies \exists! h_2 \in T_2 \colon gh^{-1}_1 \in U_2 h_2\\
			\implies gh^{-1}_1 h_2^{-1} \in U_2 \implies \exists! h_3 \dots\\
			\implies gh^{-1}_1 h_2^{-1} \dots h_r^{-1} \in U_r = \set{e}\\
			\implies g = h_r h_{r-1}\cdot \dots \cdot h_2\cdot h_1
		\end{align*}
		Eindeutigkeit der Darstellung folgt aus der Eindeutigkeit der Repäsentaten (der Nebenklassen).
		\item zu \cref{sec3:def:unique_image}:
		\begin{align*}
			(a_1^g, \dots, a_r^g) = (a_1^{g'}, \dots, a_r^{g'}) \implies a_i^{gg^{-1}} = a_i \quad \text{ Fixpunkte}
			\intertext{d.h.}
			gg^{-1} \in G_{a_1 a_2 \dots a_r} = \set{e} \implies g = g'
		\end{align*}
	\end{itemize}
\end{proof}
\begin{example}
	Sei $G = S_{\ul{4}}, M = \set{a_1, a_2, a_3, a_4} = \set{1,2,3,4}$, $G_1 \cong S_3$, $G_{1,2} \cong S_2$ und $G_{1,2,3} = \set{e}$, dann muss man etwas rechnen und bekommt
	\begin{align*}
		\begin{matrix}
			T_1 = \set{e,g_1, g_1^2, g_1^3} & \text{für } g_1=(1234)\\
			T_2 = \set{e,g_2, g_2^2} & \text{für } g_2=(234)\\
			T_1 = \set{e,g_3} & \text{für } g_3=(34)\\
		\end{matrix}
	\end{align*}
	Dann folgt mit \cref{sec3:prop3:unique_simsform}: Jedes $g \in S_{\ul{4}}$ ist eindeutig in der Form
	\begin{align*}
		g=g_3^{\alpha_3}g_2^{\alpha_2}g_1{\alpha_1}
	\end{align*}
	wobei $\alpha_3 \in \set{0,1}$, $\alpha_2 \in \set{0,1,2}$, $\alpha_1 = \set{0,1,2,3}$ und $g_0 = e$.
\end{example}
\begin{*remark}
	Speicheraufwand (in Bit):
	\begin{align*}
		\abs{T_1} = 4 &\implies 2 \text{ Bit}\\
		\abs{T_2} = 3 &\implies 2 \text{ Bit}\\
		\abs{T_3} = 2 &\implies 1 \text{ Bit}\\
				      &\implies 5 \text{ Bit}
	\end{align*}
	ist optimal, da wir $2^4 (16)\lneqq 4! (24) \le 2^5 (32)$ haben.
\end{*remark}
\begin{conclusion}[Test $g \in G$, vergleiche Probleme (P1) vom Anfang des Kapitels]
	Für $G \le S_M$ seinen eine SIMS-Basis $(a_1, \dots, a_r) \nd T_1, \dots, T_r$ bekannt (vergleiche \cref{sec3:def:sims_stuff}), $g \in S_M$ gegeben.\\
	Algorithmus zum Testen, ob $g \in G$:
	\begin{align*}
		\begin{matrix}
			\exists h_1 \in T_1 \colon a_1^{gh_1^{-1}} = a_1? & \xrightarrow{\nein} g \notin G\\
			\downarrow \ja& \\
			\exists h_2 \in T_2 \colon a_2^{gh_1^{-1}h_2^{-1}} = a_2? & \xrightarrow{\nein} g \notin G\\
			\downarrow \ja & \\
			\dots & \\
			\exists h_r \in T_r \colon a_r^{gh_1^{-1}h_2^{-1}\cdots h_r^{-1}} = a_r? & \xrightarrow{\nein} g \notin G\\
			\downarrow \ja & \\
			\exists gh_1^{-1}h_2^{-1}\dots h_r^{-1} = e? & \xrightarrow{\nein} g \notin G\\
			\downarrow \ja & \\
			g \in G
		\end{matrix}
	\end{align*}
\end{conclusion}
\begin{proof}
	1. Schritt: Wegen $G = \bigcup_{h \in T_1} G_{a_1}h$ folgt\\
	\begin{align*}
		g \in G \equival \exists h \in T_1 \colon g \in G_{a_1}h \equival \exists h \in T_1 \colon gh^{-1} \in G_{a_1}\\
		\equival \exists h \in T_1 \colon a_1^{gh^{-1}} = a_1 \wedge gh^{-1} \in G \implies \exists h \in T_1 \colon a_1^{gh^{-1}} = a_1.
	\end{align*}
	Also führt der ($\xrightarrow{\nein}$)-Zweig zu $g \notin G$. (Die weiteren Schritte sind analog.)
\end{proof}
Problem: Wie findet man das Repräsentatensystem $T_1, \dots, T_r$ für die Untergruppen $U_1, \dots, U_r$, falls Erzeugendensystem $U$ gegeben
\begin{align*}
	G = \gen{U} \quad \text{ vergleiche (P3) unter \cref{sec3:def:generatingset}}
\end{align*}
(beachte, dass $T_1\dcup \dots T_r$ ist Erzeugendensystem für $U_{1\dots r}$)\\
Antwort: Resultat von \person{Schreier} (Otto Schreier 1901 -1929)
\begin{proposition}[\person{Schreirer}-Lemma]
	\label{sec3:prop:schreier} Sei $G = \gen{U}$ mit $U = \set{g_1, \dots,g_n}$ (bzw. für endlich erzeugbare Gruppen $G$, also $\abs{G} = \infty$ sein.) $V \le G$ Untergruppe mit
	\begin{align*}
		G = Vh_1 \cup Vh_2 \cup \cdots \cup Vh_s \quad s\text{-cosets und oBdA }h_1 = e
	\end{align*}
	$T = \set{h_1, \dots, h_s}$ Repräsentatensystem für $G/V$. Für $g \in G$ sei $\phi(g) \in T$ der Repräsentant der Nebenklasse $Vg$ (d.h. $g \in Vg = V_{\phi(g)}$) Dann ist
	\begin{align*}
		X := \set{h_i g_j^{\scyllanote{k}}\phi(h_i g_j^{\scyllanote{k}})^{-1} \mid i \in \set{1,\dots,s}, g \in \set{1, \dots,m}}
	\end{align*}
	($\scyllanote{k \in \set{1,-1}}$ bei unendlichen Gruppen) ein Erzeugendensystem für die Untergruppe $V$.
\end{proposition}
\begin{*remark}
	\begin{align*}
		\scyllanote{k = \set{1,-1} \quad \text{$\;^{-1}$ ist
		Inverse}}\label{sec3:remark:infty_groups}\tag{1}
	\end{align*}
\end{*remark}
\begin{proof}
	\begin{enumerate}
		\item $X \subseteqq V$, dann
		\begin{align*}
			h_i g_j \in V\phi(h_i g_j) \implies h_i g_j \cdot \phi(h_i g_j)^{-1}
		\end{align*}
		\item $X$ ist Erzeugendensystem von $V$. Sei $g \in V \le G = \gen{U}$. Dann existiert Darstellung $g = g_{i_1}\cdots g_{i_t}$ (bei unendlichen Gruppen braucht man wieder $g^{-1}$ \scyllanote{vergleiche \eqref{sec3:remark:infty_groups}})
		\begin{align*}
			&\implies g = h_1 g_{i_1} \cdots g_{i_t} = \underbracket{h_1 g \phi(h_1 g_j)^{-1}}_{\ni X}\underbracket{\phi(h_1 g_1)}_{\exists \phi(g_1)h_{i_1}} g_{i_1}\dots g_{i_t}\\
			&= \underbracket{\cdots}_{\in X}\underbracket{h_{j_1}g_{i_2}\phi(h_{j_1}g_{i_2})^{-1}}_{\in X}\underbracket{\phi(h_{j_1}g_{i_2})}_{\exists j_2 \colon \phi(h_{j_2}g_{i_2}) = h_{j_2}}g_{i_3} \cdots g_{i_t}\\
			\overset{\text{usw.}}&{=}\underbracket{\qquad}_{\in X}\underbracket{\qquad}_{\in X}\dots\underbracket{\qquad}_{\in X}\underbracket{\phi \dots \phi^{-1}}_{\in X} \quad g \in V\\
			\intertext{Sest noch einen Schritt}
			&\implies \phi(hg) \in V 
		\end{align*}
		konkret folgt also, dass $g \in \gen{X}$.
	\end{enumerate}
\end{proof}
\begin{example}
	Sei $G := S_{\ul{n}}, V := A_{\ul{n}}$ (alternierende Gruppe der geraden Permutationen). Erzeugendensystem $g_1 := (12), g_2 := (12\dots n)$ (vergleiche \cref{sec3:remark:genSm_d}):
	\begin{align*}
		G = \gen{g_1, g_2}_{S_{\ul{n}}}
	\end{align*}
	Nebenklassenzerlegung mit Repräsentaten $h_1 :=e = (1), h_2 := (12)$:
	\begin{align*}
		S_{\ul{n}} = Vh_1 \dcup Vh_2 = A_{\ul{n}} \dcup A_{\ul{n}}(12)
	\end{align*}
	$\xRightarrow{\cref{sec3:prop:schreier}} A_{\ul{n}}$ wird erzeugt von
	\begin{align*}
		h_1 g_1 \phi(h_1 g_1)^{-1} &= e \cdot (12) \cdot (12) = e\\
		h_1 g_2 \phi(h_1 g_2)^{-1} &= \begin{cases}
			e \cdot (12\dots n) \cdot (12) = (23\dots n) &\quad \text{falls $n$ gerade}\\
			e \cdot (12\dots n) \cdot e = (213\dots n) &\quad \text{falls $n$ ungerade}
		\end{cases}\\
		h_2 g_1 \phi(h_2 g_1)^{-1} &= (12) \cdot (12) \cdot e = e\\
		h_2 g_2 \phi(h_2 g_2)^{-1} &= \begin{cases}
			(12) \cdot (12\dots n) \cdot (12) = (213\dots n) &\quad \text{falls $n-1$ gerade}\\
			(12) \cdot (12\dots n) \cdot e = (134\dots n) &\quad \text{falls $n-1$ ungerade}
		\end{cases}
	\end{align*}
	(wobei $(12) \cdot (12\dots n) = (134 \dots n)$). Also erhält man folgendes Erzeugendensystem für $A_{\ul{n}}$:
	\begin{align*}
		A_{\ul{n}} = \begin{cases}
			\gen{(234\dots n),(134\dots n)}_{S_{\ul{n}}} &\quad \text{falls $n$ gerade}\\
			\gen{(123\dots n),(213\dots n)}_{S_{\ul{n}}} &\quad \text{falls $n$ ungerade}
		\end{cases}
	\end{align*}
\end{example}
\begin{remark}
	\label{sec3:remark:genSm} Erzeugendensysteme der $S_{\ul{m}}$: folgende Mengen erzeugen $S_{\ul{m}}$:
	\begin{remarkenum}
		\item $\set{(ij)\mid i,j \in N}$ alle Transpositionen \label{sec3:remark:genSm_a}
		\item $\set{(12), (23), \dots, (n-1,n)}$ spezielle Transposotionen \label{sec3:remark:genSm_b}
		\item $\set{(12),(13),(14), \dots, (1n)}$ spezielle Transpositionen \label{sec3:remark:genSm_c}
		\item $\set{(12),(12\dots n)}$ \label{sec3:remark:genSm_d}
	\end{remarkenum}
\end{remark}
\begin{proof}[\cref{sec3:remark:genSm}]
	\item zu \cref{sec3:remark:genSm_a}: Für Zyklen gilt:
	\begin{align*}
		(a_1 \dots a_k) = (a_1 a_2)(a_1 a_3)\dots (a_1 a_k) \quad \text{ SeSt!}		
	\end{align*}
	Jede Permutation ist Produkt von Zyklen (siehe Geometrie Fehm).
	\item \cref{sec3:remark:genSm_b} - \cref{sec3:remark:genSm_b} SeSt!
\end{proof}
\begin{*remark}
	Zerlegung in Transpositionen nicht eindeutig (im Gegensatz zu SIMS-Ketten-Zerlegung \cref{sec3:prop:sims_chain_unique}), aber es gibt gewisse Invarianten
\end{*remark}
%TODO Could be something else than a definition, maybe a def and some kind of props of this /sgn-function
\begin{definition}[gerade und ungerade Permutationen]
	Sei $g \in S_{\ul{n}}$. Eine \begriff{Inversion} von $g$ ist ein Paar $(i,j)$ mit $1\le j \le n$ und $i^g > j^g$. zum Beispiel
	\begin{align*}
		\begin{pmatrix}
			1 & 2 & 3 & 4\\
			2 & 1 & 4 & 3
		\end{pmatrix} = (12)(34)
	\end{align*}
	hat zwei Inversionen: $1 < 2$, $3 < 4$.
	\begin{align*}
		\begin{pmatrix}
			1 & 2 & 3\\
			3 & 2 & 1
		\end{pmatrix} = (13)(2)
	\end{align*}
	hat die Inversionen: $1 < 2$, $2 < 3$, $1 < 3$.
	(=Anzahl der Vertauschungen von Nachbarn in der zweiten Zeile, um daraus 1. Zeile zu bekommen)
	Definiere \begriff{Signum}:
	\begin{align*}
		\sgn(g) := \begin{cases}
			1 &\quad \text{ falls \# Inversionen gerade}\\
			-1 &\quad \text{ falls \# Inversionen ungerade}\\
		\end{cases}
	\end{align*}
	$g$ heißt dann \begriff{gerade} bzw. \begriff{ungerade Permutation}.\\
	Es gilt:
	\begin{defenum}
		\item \begin{align*}
			\sgn(g) = \prod_{\stackrel{i<j}{i,j \in \ul{n}}} \frac{j^g - i^g}{j-1} = \prod_{i<j} \frac{j^{gh}-i^{gh}}{j^h - i^h}
		\end{align*}
		(für beliebige Permutation $h$)
		\begin{proof}
			Jede Inversion liefert (-1) im Zähler. Beim Aufsetzen einer Permutation $h$ wird genau dann im Produkt der Zähler ein Vorzeichen geändert, wenn auch Änderung im Nenner.
		\end{proof}
		\item $\sgn(gh) = \sgn(g) \cdot \sgn(h)$, denn
		\begin{align*}
			\sgn(g)\cdot \sgn(h) &= \prod_{i<j}\frac{j^g-i^g}{j-i}\prod_{i<j}\frac{j^h - i^h}{j-i}\\
			&= \frac{j^{gh} - i^{gh}}{j^h - i^h} \cdot \frac{j^h - i^h}{j-1}\\
			&= \frac{j^{gh}-i^{gh}}{j-1}\\
			&= \sgn(gh)
		\end{align*} \label{sec3:def:oddevenperm_b}
		\item $\sgn(e) = 1, \sgn(g^{-1}) = \sgn(g)$ (denn $1 = \sgn(g\cdot g^{1}) = \sgn(g)\cdot \sgn(g^{-1})$) \label{sec3:def:oddevenperm_c}
		\item $\sgn \colon S_{\ul{n}} \to \set{1,-1}$ ist Homomorphismus auf der multiplikativen Gruppe $\gen{\set{1,-1}, \cdot}$
		\item Die geraden Permutationen bilden Untergruppe (wegen \cref{sec3:def:oddevenperm_b}, \cref{sec3:def:oddevenperm_c} =: die alternierende Gruppe $A_{\ul{n}}$ von $S_{\ul{n}}$) \label{sec3:def:oddevenperm_e}
		\item $g \in S_{\ul{n}}$ gerade (bzw. ungerade) $\equival$ Für jede Darstellung $g = t_1,\dots,t_q$ als Produkt von Transpositionen ist $g$ gerade (bzw. ungerade)\label{sec3:def:oddevenperm_f}
		\begin{proof}[\cref{sec3:def:oddevenperm_f}]
			\begin{align*}
				g &= t_1\dots t_q\\
				e &= t_1 \dots t_q \dots t_1 = s\cdot t_1\dots t_q\\
				e &= s\cdot t_1\dots t_q = s(-1)\cdot t_{q-1}\dots t_1\\
				e &= s(-1)\cdot t_{q-1}\dots t_1 = s(-1)^q = 1
			\end{align*}
		\end{proof}
	\end{defenum}
\end{definition}
Daraus folgt:
\begin{proposition}
	Die alternierende Gruppe $A_{\ul{n}} \le S_{\ul{n}}$ besteht aus allen Permutation, die sich als Produkt einer geraden Anzahl von Transpositionen darstellen lassen.\\
	$A_{\ul{n}}$ ist Normalteiler in $S_{\ul{n}}$ und enthält $\sfrac{n!}{2}$ Elemente.
\end{proposition}
\begin{proof}
	\begin{itemize}
		\item erster Teil: \cref{sec3:def:oddevenperm_f}
		\item zweiter Teil: Homorphiesatz für $\sgn \colon S_{\ul{n}} \to \set{1,-1}$ (vergleiche \cref{sec3:def:oddevenperm_e})\\
		\begin{align*}
			\ker (\sgn) &= \set{g \mid \sgn(g) = 1} = A_{\ul{n}}\\
			\abs{S_{\ul{n}/A_{\ul{n}}}} &= \frac{\abs{S_{\ul{n}}}}{\abs{A_{\ul{n}}}} = 2
		\end{align*}
	\end{itemize}
\end{proof}
\begin{remark}
	$A_{\ul{n}}$ ist einfach für $n \le 5$ (keine Normalteiler). Daraus folgt direkt $S_n$ ist für $n \ge 5$ nicht auflösbar. (siehe Fehm Geometrie Skript)
\end{remark}