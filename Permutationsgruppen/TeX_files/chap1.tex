% !TeX spellcheck = en_US
%\section{29 October 2020}
%TODO fix numering in this chapter!
mathematische Strukturen:\\
\begin{center}
	% https://tikzcd.yichuanshen.de/#N4Igdg9gJgpgziAXAbVABwnAlgFyxMJZARgBoAGAXVJADcBDAGwFcYkQBZenACwDoABAGUcAJ2YBrHM1ExCAX1LpMufIRTlSxanSat2TAOaCR4iSEXLseAkTLaaDFm0QhZjbnwD0AeVFQ4MUkLJRAMazUiACYtHSd9VxwINBMg80swlRt1ZE0ouL0XTjlDGABqHzQQq1VbDVIAZgLndgAvPgAhAQBxcTQ0NnkdGChShBRQADNRCABbJE0QJKQojOm5hZplxGI1mfnERe2GvY2drYgkABZTg6uLpABWIfkgA
	\begin{tikzcd}
		& \text{math. Strukturen} \arrow[d] \arrow[ld] \arrow[rd] &            \\
		\text{alg. Struk} \arrow[d] & \text{relat./Ordstruk}                                  & \text{top. Struk} \\
		\text{Menge+Op.} \arrow[d]   &                                                  &            \\
		\text{z.B Gruppe}           &                                                  &           
	\end{tikzcd}
\end{center}
\begin{definition}
	Gruppe $<G,\cdot, ^{-1},e>$
	\begin{itemize}
		\item $G$ Gruppe
		\item $e$ neutrales Element zu $x$
		\item $x^{-1}$ inverses Element zu $x$
		\item $x\cdot y$ (meist $xy$) ``Produkt'', binäre Operation
	\end{itemize}
	Axiome:
	\begin{align*}
		\forall x \in G \colon& ex = xe = x\\
		\forall x \in G \colon& x x^{-1} = x^{-1}x = e\\
		\forall x,y,z \in G \colon& x(yz) = (xy)z
	\end{align*}
\end{definition}
\begin{example}
	\begin{enumerate} %TODO probably minipage here?
		\item Symmetriegruppen (geom.):	isometrische Abbildung der Ebene\\
		Drehungen: 0°, 120°, 240°\\
		Spiegelung: $s_0,s_1,s_2$\\
		Also 6 Symmetrieoperationen ($\cong$ volle symmetrische Gruppe $S_3$)
		\begin{align*}
			\chemfig{1*3(-2-3-)}
		\end{align*}
		\item Automorphismen (algebraisch bzw. kombinatorisch): Dreieck als Graph betrachten Automorphismen (bijektive Abbdildung, die Kanten in Kanten überführen). Also $\Aut(\text{Dreieck}) \cong S_3$
	\end{enumerate} %TODO chemfigs still to do!!!
	%		\begin{picture}(width,height)
	%			content...
	%		\end{picture}
	\chemfig{H(-[:270]H)}
	\chemfig{C(-[:0]H)(-[:90]H)(-[:180]H)(-[:270]H)C(-[:0]H)(-[:90]H)(-[:180]H)(-[:270]H)}
	\begin{itemize}
		\item Spiegelungen an horizontalen Achse
		\item Es gibt zahlreiche Automorphismen (Fixpunkte: 0,1,2,3,4) viele Vertauschungen
		\begin{align*}
			\abs{S_2 \times S_2 \times S_3} = 24
		\end{align*}
	\end{itemize}
	%		\begin{picture}(width,height)
	%			content...
	%		\end{picture}
	$\Gamma,\Gamma'$ sind ``im Prinzip'' das gleiche. $\exists$ Isomorphismus $f\colon \Gamma \to \Gamma'$. Das führt auf ein Isomorphieproblem: Wann sind zwei Strukturen isomorph?
\end{example}
\begin{remark}
	Isomorphieproblem zurückführbar auf Bestimmung von Automorphismen:
	\begin{align*}
		\text{als Automrophismus von $\Gamma \cup \Gamma'$}\begin{cases}
			f&\colon \Gamma \to \Gamma'\\
			f&\colon \Gamma' \to \Gamma 
		\end{cases}
	\end{align*}
	Spezialität der Symmetrien bzw. Automorphismen: haben innere Struktur sind bijektive Abbildungen (=: Permutationen) $\implies$ Permutationsgruppen.
\end{remark}
\begin{example}
	zum Isomorphieproblem\\
	chemische Isomere: Wieviel verschiedene Alkohole (Propanole, d.h. Bindungsgruppen \ch{O-H}) mit Strukturformel \ch{C3H7OH} gibt es?\\
	Antwort: $\Gamma$ siehe oben. und
	%		\begin{picture}(width,height)
	%			content...
	%		\end{picture}
	$\Gamma^{''}$, wobei $\Gamma$'s Siedepunkt = 97,1°C und $\Gamma^{''}$ = 82,4°C.
\end{example}
\begin{remark}
	gleiche Summenformel:
	%		\begin{picture}(width,height)
	%			content...
	%		\end{picture}
\end{remark}
Im Allgemeinen Lösung: Anzahl lässt sich als die Anzahl der sogenannten ``Bahnen'' (eng. Orbit) einer Permutationsgruppe beschrieben. (bestimmbar mit Lemma von \person{Cauchy}-\person{Frobenius}-Burnside). $\implies$ Abzählbartheoreme (\person{Polya}).
\begin{example}
	anderes Beispiel für Polyasche Abzählbartheorie:\\
	Wieviele wesentlich verschiedene Ketten mit 3 Sorten von Perlen gibt es? ($n_i$ Perlen der Sorte $i = 1,2,3$)
\end{example}
\begin{itemize}
	\item Permutationsgruppen sind spezielle Gruppen und trotzdem ``mehr'' als Gruppen
	\item Automorphismengruppen besonders wichtige (von algebraische Struktur)
\end{itemize}
Gruppentheoretisch ist aus den Permutationsgruppen entstanden (\person{Galois}theorie) Galoisgruppe = Permutationsgruppe
\begin{example}
	für Permuationen
	\begin{itemize}
		\item $S_n$ volle symmetrische aller Permutation auf $n$-elementiger Menge
		\item lin. Abbildung eines Vektorraumes
		\begin{align*}
			x \mapsto Ax + b
		\end{align*}
		wobei $A$ invertierbare Matrix ist
	\end{itemize}
\end{example}
Vorlesungsinahlt
\begin{itemize} 
	\item Permutations- und Gruppenwirkung
	\item Konstruktionen mt Permutationsgruppen
	\item Polyasche Abzählungstheoreme
	\item Automorphismengruppen von Relationen
\end{itemize}