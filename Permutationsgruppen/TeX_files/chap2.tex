\section{Permutationen und Permutationsgruppen}
Permutationen können unterschiedlich definiert und dargestellt werden:
\begin{enumerate}
	\item als Lineare Anordnung von Elementen einer Menge, z.B. $M = \set{a,b,c}$
	\begin{align*}
		\pi_1\colon abc &\qquad \pi_2 \colon acb\\
		\pi_3 \colon bac &\qquad \pi_4 \colon bca\\
		\pi_5 \colon cab &\qquad \pi_6\colon cba
	\end{align*}
	\item als bijektive Abbildung (in 2-Zeilen-Darstellung)
	\begin{align*}
		\pi_1 = \begin{pmatrix}
			a & b & c\\
			a & b & c
		\end{pmatrix},\;
	\pi_2 = \begin{pmatrix}
		a & b & c\\
		a & c & b
	\end{pmatrix}, \dots, \pi_6 = ...
	\end{align*}
	allgemein
	\begin{align*}
		\pi = \begin{pmatrix}
			a_1 & \dots &a_n\\
			\vdots & & \vdots\\
			a_{1i} & \dots & a_{1n}
		\end{pmatrix}
	\end{align*}
	bezeichnen. Also Abbildung $\pi \colon M \to M$ mit $a_k \mapsto a_{i_k}$, wobei $M = \set{a_1, \dots, a_n}$ Reihenfolge der Spalten spielt keine Rolle.
\end{enumerate}
\begin{definition}
	Eine \begriff{Permutation} auf Menge $M$ ist bejektiv Abbildung $f \colon M \to M$
	\begin{align*}
		S_m := S(M) := \text{Menge aller Permutationen auf }M
	\end{align*}
	Bezeichnung: für Bild $f(a)$ eines Elementes $a \in M$ $a^f$ also ist
	\begin{align*}
		f = \begin{pmatrix}
			a_1 & \dots & a_n\\
			a^f_1 &\dots & a^f_n
		\end{pmatrix}
	\end{align*}
\end{definition}
\begin{proposition}
	Für $\abs{M} = n$ gibt $n!$ viele Permutationen auf $M$.
	\begin{align*}
		\abs{S_M} = n!
	\end{align*}
\end{proposition}
\begin{proof}
	Selbststudium!
\end{proof}
\begin{definition}
	Der Graph einer Permutation
	\begin{itemize}
		\item $f\colon M \to M$
		\item $f^{\cdot} = \set{(a,b) \in M^2 \mid a^f = b}$ \begriff{Graph} von $f$, Paare $(a,b)$ als gerichtete Kanten von $a$ nach $b$ zeichnen.
		\item Graph $f^{\cdot}$ (genauer: $(M,f^{\cdot})$ hat Knotenpunktmenge $M$ und Knotenmenge $f^{\cdot}$
	\end{itemize}
\end{definition}
\begin{example}
	\begin{align*}
		f = \begin{pmatrix}
			1 & 2 & 3 & 4 & 5 & 6 & 7 & 8 & 9 & 10 & 11 & 12 & 13\\
			3 & 5 & 7 & 12 & 2 & 9 & 4 & 11 & 8 & 1 & 6 & 12 & 10
		\end{pmatrix}
	\end{align*}
	\chemfig{1*6(-3-7-4-13-10-)}\\
	\chemfig{6*4(-9-8-11-)}
	\[
		% https://tikzcd.yichuanshen.de/#N4Igdg9gJgpgziAXAbVABwnAlgFyxMJZABgBpiBdUkANwEMAbAVxiRACYQBfU9TXfIRQBGclVqMWbAKzdxMKAHN4RUADMAThAC2SMiBwQkokACMYYKEgDMxHuq27EJw3urnLSACwBOLhS4gA
		\begin{tikzcd}
			2 \arrow[r, bend left] & 5 \arrow[l, bend left=49]
		\end{tikzcd}
		% https://tikzcd.yichuanshen.de/#N4Igdg9gJgpgziAXAbVABwnAlgFyxMJZABgBpiBdUkANwEMAbAVxiRAEYAmEAXypBhQA5vCKgAZgCcIAWyRkQOCPOoMIENETLjGcGPwZ0ARjAYAFTLnyFEISViEALHLwo8gA
		\begin{tikzcd}
		12 \arrow[loop, distance=2em, in=305, out=235]
		\end{tikzcd}
	\]
Fakt Vorraussetzung $M$ endlich: Graph $f^{\cdot}$ einer Permutation $f$ ist ein Kreis (Zyklus) oder die Vereinigung  von paarweisen disjunkten Kreisen (Zyklen) (folgt aus Bijektivität) (Gilt nicht für unendliche Mengen $f \colon \Z \to \Z \mit x \mapsto x+1$)
\end{example}
Ab jetzt $M$ endlich.
\begin{definition}
	Die \begriff{Zyklendarstellung einer Permutation $f$} (entspricht ``lineares Aufschreiben von $f^{\cdot}$'')
\end{definition}
\begin{example}
	$f$ wie oben
	\begin{align*}
		(1,3,7,4,13, 10)(2,5)(6,9,8,11)(12)	
	\end{align*}
\end{example}
Falls $M$ fest, Zyklen der Menge 1 weglassen, das nennt man die \begriff{verkürtzte Zyklendarstellung}. \begriff{zyklische Permutation} := Permutation mit genau einem Zyklus in der verkürtzten Zyklendarstellung. Die identische Permutation $x \mapsto x$, Zyklendarstellung $(1)$.\\
Beachte: $(abc),(bca),(cab)$ bezeichnen die selbe Permutation. (nur Reihenfolge, nicht Anfangselement wichtig)
\begin{definition}
	Die \begriff{Multipllikation} (Produkt) von Permutationen ist die Hintereinanderausführung von Abbildungen.
	\[
	% https://tikzcd.yichuanshen.de/#N4Igdg9gJgpgziAXAbVABwnAlgFyxMJZABgBpiBdUkANwEMAbAVxiRAFkQBfU9TXfIRQBGclVqMWbTjz7Y8BIgCYx1es1aIO3XiAzzBRMsPHqpWujrkDFI0ibWTNIOgD0AZlb38FQ5CocJDTY3YHcAcy5ucRgocPgiUHcAJwgAWyQyEBwIJFEQBjoAIxgGAAUfQy0GGHccEEdgrU9ZEBT0vOocpBUg8xBwr3aMxABmLtzEABYuuiwGNjS6NDhu1uGkGezJgFZZ+cXl1dyuCi4gA
	\begin{tikzcd}
		M \arrow[r, "f"]     & M \arrow[r, "g"]       & M      \\
		a \arrow[r, maps to, "f"] & a^f \arrow[r, maps to, "g"] & a^{fg}
	\end{tikzcd}
	\]
	Produkt $fg$ (oder $f;g$ oder $f\cdot g$, oder auch $g \circ g$) wird definiert durch 
	\begin{align*}
		a^{fg}:= (a^f)^g
	\end{align*} 
	ist wieder Permutation.
\end{definition}
\begin{example}
	\begin{align*}
		\begin{pmatrix}
			1 & 2 &3\\
			2 & 1 &3
		\end{pmatrix}
		\begin{pmatrix}
			1 & 2 & 3\\
			3 & 2 & 1
		\end{pmatrix}=
		\begin{pmatrix}
			1 & 2 & 3\\
			2 & 3 & 1
		\end{pmatrix}
	\end{align*}
	\begin{itemize}
		\item Zykeldarstellung: $(12)(3)\cdot (13)(2) = (123)$
		\item verkürtzt: $(12)\cdot(13) = (123)$ 
	\end{itemize}
\end{example}
Fakt: verkürtzte Zyklendarstellung $k$ Zyklen
\begin{align*}
	f = (--c_1--)(--c_2--)\dots (--c_3--)\\
	g = (--c_i--)
\end{align*}
$g$ also Permutation zyklisch.
Also haben wir
\begin{align*}
	f = g_1 \cdot g_2 \cdot \cdots \cdot g_k
\end{align*}
(Komposition)
\begin{proposition}
	Die Permutationen aus $S_M$ bilden mit der Multiplikation eine Gruppe. die volle symmetrische Gruppe vom Grad $\abs{M}$. (Einselement $(1)$ = identische Abbildung, inverse Abbildung $f^{-1}$ Zeilen in 2-Zeilendarstellung vertauschen. $(ab \dots xy)^{-1} = (yx\dots ba)$.
\end{proposition}
\begin{proof}
	\sest
\end{proof}
\begin{remark}
	Alle gruppentheoretische Begriffe sind auch für Permutationsgruppen definiert:Ordnung $\ord(f) = \min\set{m \mid f^m = e}$, Untergruppen, Normalteiler, konjugierte Elemente, usw.
\end{remark}
% 2nd lecture 3.11.2020
\begin{definition}
	Eine \begriff{Permutatiosngruppe} $G$ von Grad $n$ ist eine Untergruppe der vollen symmetrischen Gruppe $S_M$ vom Grad $n$.\\
	Bezeichnung: $(G,M)$ oder auch falls $G$ Untergruppe ist, $G \le S_M$. Wobei meist
	\begin{align*}
		M &= \set{0,1,\dots,n-1} =: n \implies S_n\\
		M &= \set{1,2,\dots, n} =: \uline{n} \implies S_{\uline{n}}
	\end{align*}
\end{definition}
\begin{remark}
	Weitere Schreibweisen für $U,V \subseteqq S_M$, dabei
	\begin{align*}
		UV &= \set{uv \mid u\in U, v \in V}
		\intertext{$a\in M, B\subseteqq M, g \in S_M$}
		a^u &= \set{a^u \mid u \in U}\\
		B^g &= \set{b^g \mid b \in B}\\
		B^u & = \set{b^u \mid b \in B, u \in U}
	\end{align*}
\end{remark}
\begin{proposition}[Gruppenkriterium]
	Sei $M$ endlich. Dann ist $U \subseteqq S_M$ Gruppe genau dann, wenn $U\cdot U \subseteqq U$.
\end{proposition}
\begin{proof}
	\sest
\end{proof}
\begin{example}
	Symmetrieabbildung des Rechtecks in der Ebene
%	\begin{center}
%		\tikzset{every picture/.style={line width=0.75pt}} %set default line width to 0.75pt        
%		
%		\begin{tikzpicture}[x=0.75pt,y=0.75pt,yscale=-1,xscale=1]
%			%uncomment if require: \path (0,300); %set diagram left start at 0, and has height of 300
%			
%			%Shape: Rectangle [id:dp8018685777285718] 
%			\draw   (100,110) -- (159.83,110) -- (159.83,169.67) -- (100,169.67) -- cycle ;
%			%Straight Lines [id:da9350361269288623] 
%			\draw  [dash pattern={on 0.84pt off 2.51pt}]  (79.83,139.67) -- (179.83,139.67) ;
%			%Straight Lines [id:da8951381985636013] 
%			\draw  [dash pattern={on 0.84pt off 2.51pt}]  (129.5,90) -- (129.83,190.67) ;
%			
%			% Text Node
%			\draw (161,95) node [anchor=north west][inner sep=0.75pt]  [font=\footnotesize] [align=left] {2};
%			% Text Node
%			\draw (161.83,172.67) node [anchor=north west][inner sep=0.75pt]  [font=\footnotesize] [align=left] {3};
%			% Text Node
%			\draw (90,97) node [anchor=north west][inner sep=0.75pt]  [font=\footnotesize] [align=left] {1};
%			% Text Node
%			\draw (88,173) node [anchor=north west][inner sep=0.75pt]  [font=\footnotesize] [align=left] {4};
%			% Text Node
%			\draw (124,70) node [anchor=north west][inner sep=0.75pt]  [color={rgb, 255:red, 245; green, 166; blue, 35 }  ,opacity=1 ] [align=left] {II};
%			% Text Node
%			\draw (182,131) node [anchor=north west][inner sep=0.75pt]  [color={rgb, 255:red, 245; green, 166; blue, 35 }  ,opacity=1 ] [align=left] {I};
%			
%			
%		\end{tikzpicture}
%		
%	\end{center}
können durch Permutation der Eckpunkte beschrieben werden. Also
\begin{center}
	\begin{tabular}{|c|c|}
		Identitätsabbildung: & $(1) =: e$\\
		Drehung 180°: & $(13)(24) =: g_1$\\
		Spiegelung an I: & $(14)(23) =: g_2$\\
		Spiegelung an II: & $(12)(34) =: g_3$  
	\end{tabular}
\end{center}
Damit ist $G = \set{e,g_1,g_2,g_3}$ Permutationsgruppe und $G \cong$ Symmetriegruppe des Rechtecks in der Ebene.
\begin{center}
	\begin{tabular}{c|cccc}
		$\cdot$ & $e$ 	& $g_1$ & $g_2$ & $g_3$\\
		\hline
		$e$   	& $e$   & $g_1$ & $g_2$ & $g_3$\\
		$g_1$ 	& $g_1$ & $e$   & $g_3$ & $g_2$\\
		$g_2$ 	& $g_2$ & $g_3$ & $e$   & $g_1$\\
		$g_3$ 	& $g_3$ & $g_2$ & $g_1$ & $e$\\
	\end{tabular}
\end{center}
das ist die \person{Klein}sche Vierergruppe, die isomorph zu $\Z_2 \times \Z_2$ ist.
\end{example}
\begin{definition}
	Sei $(G,M)$ Permutationsgruppe, $a\in M$, dann definiere
	\begin{enumerate}
		\item Den \begriff{Stabilisator} von $a$
		\begin{align*}
			G_a := \set{g \in G \mid a^g = a}
		\end{align*}
		Allgemeiner haben wir 
		\begin{align*}
			G_{a_1, a_2, \dots, a_m} := \bigcap_{i=1}^m G_{a_i}.
		\end{align*}
		\item Die \begriff{Bahn} (1-Bahn, Orbit)
		\begin{align*}
			a^G := \set{a^g \mid g \in G}.
		\end{align*}
		Also ist der 1-$\Orb(G,M):=$ Menge aller 1-Bahnen (Andere Bezeichnung: $G\Vert M$).
		\item $B \subseteq M$ \begriff{invariante Menge} (bezüglich $G$) $:\equival B^G \subseteqq B$ 
		\item $G$ \begriff{transitiv} $\equival \exists a \in M \colon a^G = M$
		\begin{*remark}
			Äquivalent dazu sind:
			\begin{align*}
				\forall a \in M \colon a^G = M\\
				\abs{1-\Orb(G,M)} = 1
			\end{align*}
		\end{*remark}
	\end{enumerate}
\end{definition}
\begin{lemma}
	Sei $G \le S_M, a \in M$. Es gilt:
	\begin{lemmaenum}
		\item $G_a$ ist Untergruppe von $G$. \label{lem_1.11_1}
		\item $G_{a^g} = g^{-1}(G_a)g$ $(g \in G)$ \label{lem_1.11_2}
		\item Durch $a \sim b \equival a^G = b^G$ ist eine Äquivalenzrelation gegeben und es gilt:
		\begin{align*}
			1-\Orb(G,M) = M/\sim
		\end{align*}
		Die menge aller 1-Bahnen bildet eine Zerlegung von $M$ (zwei Bahnen sind gleich oder disjunkt).\\
		Beachte:
		\begin{align*}
			b \in a^G \equival a \in b^G \quad \text{ Sest!}
		\end{align*}
		\label{lem_1.11_3}
		\item Jede invariante Menge $B \subseteqq M$ ist Vereinigung von 1-Bahnen
		\begin{align*}
			B = B^G = \bigcup_{b \in B} b^G
		\end{align*}
		\label{lem_1.11_4}
	\end{lemmaenum}
\end{lemma}
\begin{proof}
	\cref{lem_1.11_1} - \cref{lem_1.11_3} SeSt, \cref{lem_1.11_4} klar nach Definition!
\end{proof}
\begin{remark}
	Repräsentatensystem einer Zerlegung $1-\Orb(G,M)$ heisst \begriff{Transversale}.
\end{remark}
\subsection*{Wiederholung Algebra}
\begin{proposition}[Satz von \person{Lagrange}]\label{prop:lagrange}
	Die Ordnung $\abs{U}$ jeder Untergruppe einer endlichen Gruppe $G$ ist Teiler der Gruppenordnung. Es gilt
	\begin{align*}
		\abs{G} = [G\colon U]\cdot \abs{U}
	\end{align*}
	(wobei $[\cdot\colon \cdot]$ der Index ist).
\end{proposition}
\begin{proof}
	Index $[G\colon U] = \abs{G/U}$ = Anzahl $k$ der (rechts-)Nebenklassen $Ug$ in der Nebenklassenzerlegung.
	\begin{align*}
		G = U g_1 \cup U g_2 \cup \dots U g_k
	\end{align*}
	Dabei ist die Nebenklasse durch $Ug := \set{u\cdot g \mid u \in U}$ und $G/U := \set{Ug \mid g \in G}$ und wegen $\abs{U} = \abs{Ug}$ folgt \cref{prop:lagrange}.
\end{proof}
