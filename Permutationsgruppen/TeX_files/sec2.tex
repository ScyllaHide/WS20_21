% !TeX spellcheck = de_DE
%10.11.2020
\begin{definition}[Permutationsdarstellung]
	\label{sec2:def:permudarstellung}
	\begin{defenum}
		\item Ein Homomorphismus
		\begin{align*}
			\psi\colon G \to S_M
		\end{align*}
		einer (abstrakten) Gruppe $G$ in eine symmetrische Gruppe $S_M$ heisst \begriff{Permutationsdarstellung} von $G$. (vom Grad $\abs{M}$)
		\item $\psi$ \begriff{true}, falls $\psi$ injektiv.
	\end{defenum}
\end{definition}
\begin{*remark}
	$\psi$ ist treu $\equival$ $\ker \psi = \set{g \mid \psi(g) = e} = \set{e}$, mithilfe des Homomorphiesatz folgt $G \cong \Im \psi(G) = \psi[G] \le S_M$. ($G$ ist ``praktisch'' Permutationsgruppe.)
\end{*remark}
\begin{definition}
	Sei $G = \gen{G, \cdot, ^{-1}, e}$ Gruppe, $M$ Menge. Eine Abbildung
	\begin{align*}
		\phi \colon \begin{cases}
			M \times G &\to M\\
			(x,g) &\mapsto xg = \phi(x,g)
		\end{cases}
	\end{align*}
	heisst \begriff{Gruppenwirkung}(alt: Gruppenoperation, eng. group action) von $G$ auf Menge $M$, falls folgendes gilt:
	\begin{enumerate}
		\item $\phi(x,e) =xe_G = x \quad \forall x \in M$
		\item 
		\begin{align*}
			(xg)g' &= x(g\cdot_G g')\\
			\phi(\phi(x,g),g') &= \phi(x,g\cdot_G g')
		\end{align*}
	\end{enumerate}
	Sprechweise: $G$ \ul{wirkt} (operiert, eng. acts) auf $M$\\
	Schreibweise: $(G,M)$
\end{definition}
\begin{*remark}
	Jede Permutationsgruppe $G \le S_M$ operiert auf natürliche Weise auf $M$
	\begin{align*}
		xg = \phi(x,g) = x^g
	\end{align*}
	(oft Schreibweise $x^g$ statt $\phi(x,g) \wedge xg$)
\end{*remark}
\begin{proposition}\label{sec2:prop:groupaction}
	Jeder Gruppenwirkung
	\begin{align*}
		\phi\colon M\times G \to M
	\end{align*}
	entspricht in ein-eindeutiger Weise einer Permutationsdarstellung.\\
	$\psi\colon G \to S_M$ und umgekehrt, und zwar gemäß
	\begin{align*}
		x^{\psi(g)} &= xg & (= \phi(x,g)\\
		(x\in M, g\in G) &:= & \text{falls $\phi$ gegeben}\\
		& := &\text{falls $\psi$ gegeben}
	\end{align*}
\end{proposition}
\begin{proof}
	\sest. (Hinweis: Es muss gezeigt werden, dass $\psi$ ein Gruppenhomomorphismus ist.)
\end{proof}
\begin{lemma}
	\begin{lemmaenum}\label{sec2:lem:representation_permugroups}
		\item Ist $G$ (abstrakte) Gruppe, so ist durch
		\begin{align*}
			h \in G\quad h^{\st{g}}
		\end{align*}
		(rechts-multiplikation mit $g$) für jedes $g \in G$ eine Permutation: $\st{g}\in S_{G}$ gegeben.\label{sec2:lem:reppermu:rimu}
		\item $\psi\colon G \to S_G$ mit $g \mapsto \st{g}$ ist Permutationsdarstellung\label{sec:lem:reppermu:rep}
		\item $\phi\colon (G\times G) \to G$ mit $(h,g) \mapsto hg$ (Produkt in Gruppe $G$) zugehörige Gruppenwirkung\label{sec2:lem:reppermu:groupact}
		\item $\phi$ ist treu (und heißt \begriff{rechtsreguläre Darstellung} von $G$) \label{sec2:lem:actionfaithful}
	\end{lemmaenum}
\end{lemma}
Folgerung aus \cref{sec2:lem:actionfaithful} (vergleiche Bemerkung zu \cref{sec2:def:permudarstellung}) ist
\begin{conclusion}[Satz von \person{Cayley}]
	Für eine beliebige Gruppe $G$ ist
	\begin{align*}
		\st{G} = \set{\st{g} \mid g \in G} \subset S_G
	\end{align*}
	eine zu $G$ isomorphe (da treu) Permutationsgruppe $(\st{G},G)$ heißt \begriff{rechtsreguläre Darstellung} von $G$.
\end{conclusion}
\begin{proof}[\cref{lem:representation_permugroups}]
	\begin{itemize}
		\item \cref{sec2:lem:reppermu:rimu} und \cref{sec:lem:reppermu:rep} folgen wegen \cref{sec2:prop:groupaction} aus 3)
		\item zu zeigen \cref{sec2:prop:groupaction} für $\phi$
		\begin{enumerate}
			\item $\phi(h,g) = he = h$
			\item $(hg)g^{-1} = h(gg^{-1})$ (assoziativ Gruppen der Gruppenmultiplikation)
		\end{enumerate}
		\item noch zu zeigen \cref{sec2:lem:reppermu:groupact}: Sei $\st{g}_1 = \st{g}_2$ (gilt unter $\psi$). Dann $h^{\st{g}_1} = h^{\st{g}_2}$ impliziert $\xRightarrow{h^{-1}} g_1 = g_2$. (Da $h^{-1} \cdot h = e$ gilt.)
	\end{itemize}
\end{proof}
\begin{example}
	Sei $G = S_3 = \set{g_1, g_2, g_3, g_4, g_5, g_6}$ mit $M = \set{1,2,3}$
	\begin{align*}
		\begin{matrix}
			g_1 = e = (1) & g_2 = (12) & g_3 = (13)\\
			g_4 = (23) & g_5 = (123)  & g_6 = (132)
		\end{matrix}
	\end{align*}
	Multiplikationstafel (Cayley table)
	%TODO fixxxx
%	\[
%	\begin{center}
%		\begin{tabular}{c|cc|c|ccc}
%			$\cdot$ & $g_1$ & $g_2$ & $g_3$ & $g_4$ & $g_5$ & $g_6$\\
%			\hline
%			$g_1$   & & & $g_3$ & & \\
%			$g_2$ 	& & & $g_5$ & & \\
%			$g_3$ 	& & & $g_1$ & & \\
%			$g_4$ 	& & & $g_6$ & & \\
%			$g_5$ 	& & & $g_2$ & & \\
%			$g_6$ 	& & & $g_4$ & & \\
%		\end{tabular}
%	\end{center}
%	\]
	then we have: $\st{g}_i g \to gg_i$ wird durch die 3. Spalte des Cayleytafel beschrieben, d.h.
	\begin{align*}
		\st{g}_3 = \begin{pmatrix}
			g_1 & g_2 & g_3 & g_4 & g_5 & g_6\\
			g_3 & g_5 & g_1 & g_6 & g_2 & g_4
		\end{pmatrix} = (g_1 g_3)(g_2 g_5)(g_3 g_1)
	\end{align*}
	Zyklenschreibweise.
\end{example}
\begin{*remark}
	\begin{enumerate}
		\item $G$ wirkt $S_{\ul{3}}$ auf $M = \set{1,2,3}$
		\item $\st{G} = \st{S}_{\uline{3}}$ wirkt auf der Menge $S_{\uline{3}} = \set{g_1, \dots, g_6}$ d.h. ist Untergruppe der $S_G$, SeSt kein $\st{g} \neq e$ hat eine Fixpunkt
		\item Jedes $\st{g}$ zerfällt in ein Produkt von Zyklen gleicher Menge $\ord(g)$, vgl. vorheriges Beispiel: $\ord(g_2) = 2$
		\item $\st{G}$ hat Grad $\abs{G}$.
		\item $\st{G}$ ist transitiv (d.h. gibt nur eine Bahn $e^{\st{G}} = G$)
		\item Die EGS 2. und 5. charakterisieren die Regularität von $\st{G}$ (vgl 5.4) 
	\end{enumerate}
\end{*remark}
\begin{example}[Weitere Beispiele von Gruppenwirkungen einer (abstrakten) Gruppe $G$]
	\begin{enumerate}
		\item Wirkung durch Konjugation (siehe Geometrie Kurs)
		\begin{align*}
			\phi \colon \begin{cases}
				G \times G &\to G\\
				(h,g) &\mapsto g^{-1}hg
			\end{cases}
		\end{align*}
		zugehörige Permutationsdarstellung:
		\begin{align*}
			\psi \colon \begin{cases}
				G &\to S_G\\
				g &\to \psi(g)
			\end{cases}\\
		h^g = h^{\psi(g)} = g^{-1}hg
		\end{align*}
	\item Wirkung auf Untergruppen $U \subseteq G$ ($\Sub(G)$ Menge der Untergruppen von $G$)
		\begin{align*}
			\psi \colon \begin{cases}
				\Sub(G) \times G &\to \Sub(G)\\
				(U,g) &\to g^{-1}Ug
			\end{cases}
		\end{align*}
	zugehörige Permutationsdarstellung
	\begin{align*}
		\phi \colon G \to S_{\Sub(G)}
	\end{align*}
	\item Wirkung auf rechte-Cosets $G/U$
	\begin{align*}
		U \subseteq G, G/U = \set{Uh \mid h \in G}\\
		\phi \colon \begin{cases}
			G/U \times G &\to G/U\\
			(Uh,g) &\mapsto Uhg
		\end{cases}\\
	\phi \colon G \to S_{G/U}
	\end{align*}
	\end{enumerate}
\end{example}
\begin{example}[Wirkungen von Permutationsgruppen $(G,M)$ auf andere Mengen]\label{sec2:exm:actions}
	\begin{expenum}
		\item induzierte Wirkung $(G,\powerset(M))$ auf Potenzmenge $\powerset(M)$
		\begin{align*}
			\powerset(M) \times G &\to \powerset(M)\\
			(B,g) &\mapsto B^g = \set{h^g \mid h \in B} 
		\end{align*}\label{sec2:exm:action_a}
		\item Einschränkung: induzierte Wirkung $(G,\powerset_m(M))$ auf $m$-elementige Teilmenge
		\begin{align*}
			\phi\colon \begin{cases}
				\powerset_m(M) \times G &\to \powerset_m(M)\\
				(B,g) &\mapsto B^g \quad \abs{B} = m
			\end{cases}
		\end{align*}
		Bezeichnung dieser Wirkung auch $(G^{m}, \powerset_m(M))$ \label{sec2:exm:action_b}
		\item induzierte Wirkung von $(G,M^m)$ auf $m$-Tupel d.h. auf $M^m$
		\begin{align*}
			\phi \colon \begin{cases}
				M^m \times G &\to M^m\\
				(a_1, \dots, a_m)^g &\mapsto (a_1^g, \dots, a_m^g)
			\end{cases}
		\end{align*}
		Bezeichnung: $(G^{[m]}, M^m)$. \label{sec2:exm:action_c}
	\end{expenum}
\end{example}