% !TeX spellcheck = en_US
\documentclass[ngerman,a4paper,order=firstname]{mathscript}
\usepackage{mathoperators}

% % % local commands
\DeclareMathOperator{\Ad}{Ad}				% Adjoint
\DeclareMathOperator{\PSL}{PSL} 			% projective linear group 
\newcommand{\with}{\text{ with }}
\newcommand{\nd}{\text{ and }}
\newcommand{\for}{\text{ for }}
\renewcommand{\rhd}{\triangleright}
\renewcommand{\lhd}{\triangleleft} 			% normal subgroups
\DeclareMathOperator{\Set}{Set}				% Category of sets
\newcommand{\cat}{\mathcal C}				% some category
\DeclareMathOperator{\Vect}{Vect}			% Category of vector spaces
\DeclareMathOperator{\Grp}{Grp}				% Category of groups
\DeclareMathOperator{\QVect}{QVect}			% Cat of quasi vector bundles
\DeclareMathOperator{\Mod}{Mod}				% Cat of moduls
\newcommand{\EE}{\mathscr E}				% some cat E
\DeclareMathOperator{\Ann}{Ann}				% annihilator
\DeclareMathOperator{\morph}{Morph}				% cat of morphs
\newcommand{\Circlearrowleft}{\mathbin{\rotatebox[origin=c]{180}{$\circlearrowright$}}}
\DeclareMathOperator{\Cl}{Cl}				% conjugation class of something.
\renewcommand{\phi}{\varphi}				% always varphi for phi
\DeclareMathOperator{\Iso}{Iso}				% set of isomorphisms
\DeclareMathOperator{\Lin}{\mathscr L}		% set of continuous linear maps from V to W
\newcommand{\hset}[3]{\mathrm{H}^{#1}(#2;#3)} % set of equivalent cocycles over X
\newcommand{\isoset}[3]{\Phi^{#1}_{#2}(#3)} % isomorphism class of - F vector bundles
\DeclareMathOperator{\Open}{Open}				% inclusion category for presheafs
\newcommand{\presheaf}{\mathcal S}			% presheaf

% % % % color note stuff
%\newcommand{\marganote}[1]{\textcolor{gray}{#1}}
%\newcommand{\fehmnote}[1]{\textcolor{red}{#1}}
\newcommand{\scyllanote}[1]{\textcolor{blue}{#1}}

% Nice looking emptyset
\let\oldemptyset\emptyset
\let\emptyset\varnothing

% get this stupid arrows:
%\usepackage{mathabx,graphicx}  % ---> add to mathoperators
%\def\Circlearrowleft{\ensuremath{%
%		\rotatebox[origin=c]{180}{$\circlearrowleft$}}}
%\def\Circlearrowright{\ensuremath{%
%		\rotatebox[origin=c]{180}{$\circlearrowright$}}}
%\def\CircleArrowleft{\ensuremath{%
%		\reflectbox{\rotatebox[origin=c]{180}{$\circlearrowleft$}}}}
%\def\CircleArrowright{\ensuremath{%
%		\reflectbox{\rotatebox[origin=c]{180}{$\circlearrowright$}}}}
%\begin{document}
%	\Huge
%	$\circlearrowleft \circlearrowright $
%	
%	$\Circlearrowleft \Circlearrowright $
%	
%	$\CircleArrowleft \CircleArrowright $

% % % local packages
\usepackage{braids}

\newlist{remarkenum}{enumerate}{1}
\setlist[remarkenum]{label=(\alph*),ref=\theremark~(\alph*)}
\crefalias{remarkenumi}{remark}

\newlist{propenum}{enumerate}{1}
\setlist[propenum]{label=(\alph*),ref=\theproposition~(\alph*)}
\crefalias{propenumi}{proposition}

\newlist{expenum}{enumerate}{1}
\setlist[expenum]{label=(\alph*),ref=\theexample~(\alph*)}
\crefalias{expenumi}{example}

\newlist{lemmaenum}{enumerate}{1}
\setlist[lemmaenum]{label=(\alph*),ref=\thelemma~(\alph*)}
\crefalias{lemmaenumi}{lemma}

\newlist{defenum}{enumerate}{1}
\setlist[defenum]{label=(\roman*),ref=\thedefinition~(\roman*)}
\crefalias{defenumi}{definition}

\newlist{theoenum}{enumerate}{1}
\setlist[theoenum]{label=(\roman*),ref=\thedefinition~(\roman*)}
\crefalias{theoenumi}{theorem}

\title{\textbf{Permutationsgruppen WS 20/21}}
\author{Prof. Pöschel}

\begin{document}
\pagenumbering{roman}
\pagestyle{plain}

\maketitle

\hypertarget{tocpage}{}
\tableofcontents
\bookmark[dest=tocpage,level=1]{Inhaltsverzeichnis}

\pagebreak
\pagenumbering{arabic}
\pagestyle{fancy}

\chapter*{Preface}
The plan is to go rather fast through the first chapter of this book(to get fast to K-Theory), take some notes, write down ideas, examples and remarks, rewrite proofs, so that i can understand them in my way. Also add sometimes reminders to concepts/definitions, so that i have a good overview about vector bundles and of course K-Theory. I will also use notation from courses i took in the past. But I will put remarks for the reader.
Hope you will find these notes helpful in any way.

ScyllaHide, \today
\chapter{Vector Bundles}
% !TeX spellcheck = en_US
\section{Topological Manifolds}
Lets recap a few definitions first, then we can define what a topological manifold is:
\begin{erinnerung}
	\begin{enumerate}
		\item topology on a set $M$ is a collection $\tau$ of  of subsets of $M$ we have the following
		\begin{enumerate}
			\item $\emptyset, M \in \tau$
			\item $\bigcup_{i=1}^{\infty} O_i \in \tau$ (infinite)
			\item $\bigcap_{i=1}^n O_i \in tau$
		\end{enumerate}
		then the pair $(M,\tau)$ is called topological space.
		\begin{itemize}
			\item Then we can define a subset $U \subset M$ is an open set if it belongs to the topology $\tau$.
			\item A neighborhood $U \in \tau$ of a point $p \in M$, contains simply $p \in M$.
			\item Closed set $F \subset M$, where the complement $M \setminus F = F^C$ is open.
			\item interior $\int A$ largest open set contained in $A$
			\item closure $\bar{A}$ smallest closed set containing $A$
			\item subset topology on $A\subset M$ is $\tau_A := (U \cap A)_{U \in \tau}$
		\end{itemize}
	\item quotient topology: quotient space $M\setminus sim$, where $\sim$ is an equivalence relation.
	define canonical projection $\pi\colon M \to M\setminus \sim$, the quotient topo is then, be letting $V \subset M\setminus \sim$ be open iff $\pi^{-1}(V) \subset M$ is open in $M$.
	\item product topology: 
	\end{enumerate}
\end{erinnerung}
\begin{definition}[topological manifold]
	M of dimension $n$ is a topological space with properties
	\begin{defenum}
		\item $M$ is Hausdorff for all $p_1\neq p_2 \in M$ there exists nbhs $V_1(p_1), V_2(p_2)$ st. $V_1 \cap V_2 = \emptyset$.\label{chap1_1:def:topomf:haus}
		\item $\forall p \in M,\phi\colon V(p) \subset M \to U_O \subset \R^n$ ($U_O$ open subset).\label{chap1_2:def:topomf:nbh}
		\item $M$ has second countability axiom, this means $M$ has countable basis for its topology.\label{chap1_2:def:topomf:basis}
	\end{defenum}
\end{definition}
\begin{example}[topological manifolds]
	\begin{expenum}
	\item $M \ubset \R^n$, then $(M,\tau_U)$ topological space, $U \subset M$ open iff $M\cap V = U$, $V_O \subset \R^n$.
	\item circle has dimension $n=1$
	\begin{align*}
		S^1 = \set{(x,y) \in \R^2 \mid x^2 + y^2 = 1},
	\end{align*}
	\cref{chap1_1:def:topomf:haus}, \cref{chap1_2:def:topomf:basis} inherited from ambient space $\R^2$. For \cref{chap1_2:def:topomf:nbh} we get a homeomorphism $\phi \colon U(p) \subset S^1 \to T \subsetset \R$, here $T$ open interval, choose the nbh $U(p)$ in such a way that the normal $n_p$ on this point is not parallel to a coordinate axis, then this projection is the needed homeomorphism. We basically bend (homeo) the nbh $U(p)$ on $S^1$, such that we have an open interval in $\R$. \scyllanote{Why we need the normal-part?}
	This can be generalized to $S^2$ (dimension $n=2$).
	\item surface of cube is topo mf its homeomorphic to $S^2$
	\end{expenum}
\end{example}
The next few examples use identification of a square.
\begin{example}
	\begin{expenum}
		\item Torus $\TT^2$ quotient of unit square $Q=[0,1]^2 \subset \R^2$ with equivalence relation(ER)
		\begin{align*}
			(x,y) \sim (x+1,y) \sim (x,y+1)
		\end{align*}
		with quotient topology.
		%TODO explain this a lil better ...
		\item\person{Klein} bottle $K^2$ quotient of $Q$ with ER
		\begin{align*}
			(x,y) \sim (x+1,y) \sim (1-x,y+1)
		\end{align*}
		\item projective place $\R P^2$ quotient of $Q$ by ER
		\begin{align*}
			(x,y)\sim (x+1,1-y) \sim (1-x,y+1)
		\end{align*}
	\end{expenum}
\end{example}
\begin{*example}
	the \person{Möbius}-strip is a topo mf with boundary.
\end{*example}
\begin{definition}[closed half space, topo mf with boundary, interior pts, bounary pts]
	%TODO here is not clear what \partial\HH^n?
\end{definition}
\section{Differential Manifolds}
%TODO
\section{Differentiable Maps}
%TODO
\section{Tangent Space}
%TODO
\section{Immersions and Embeddings}
%TODO
\section{Vectors Fields}
%TODO
An example to understand the interaction between vector fields, 1-parameter diffeomorphism group and local flow.
\begin{*example}
	Haben wir ein Vektorfeld $X$ auf $M$, für dass global ein Fluss $F:M \times (-\epsilon,\epsilon)\to M$ existiert, so gilt natürlich $F(x,0)=x$, also $F(0,\bullet)=\id_M$ ist ein Diffeomorphismus von $M$. Daher gilt auch auf einer kleinen Umgebung $U$ von 0, dass $F(x,u)$ für $u$ in $U$ ein Diffeomorphismus auf $M$ ist. Dann sieht man aber leicht ein, dass jedes $F(t,\bullet)$ als Komposition endlich vieler Faktoren $F(u,\bullet)$ geschrieben werden (mit $u$ in $U$). Daher ist $F(t,\bullet)$ für alle $t$ definiert und für alle $t$ ein Diffeomorphismus auf $M$. Wir erhalten also einen Homomorphismus $\R\to\Diffeo(M)$.\\
	Die Diffeomorphismengruppe $Diff(M)$ einer Mannigfaltigkeit ist einfach die Gruppe aller Diffeomorphismen $f\colon M\to M$ mit Komposition. $Diff(M)$ wirkt auf $M$.\\
	Ein einfaches Beispiel einer 1-Parameter-Untergruppe einer Diffeomorphismengruppe ist das folgende: Sei $M=\R$ und $X=1$ das Vektorfeld, das konstant 1 ist. Dann ist der zugehörige Fluss zu $X$: $F(x,t)=x+t$. Man sieht leicht, dass $F(F(x,t),s)=F(x,t+s)$.
\end{*example}
\section{Lie Groups}
%TODO
\section{Orientability}
%TODO
\section{Manifolds with Boundary}
%TODO


\part*{Anhang}
\addcontentsline{toc}{part}{Anhang}
\appendix

\nocite{*}
%\bibliography{literatur}
%\bibliographystyle{acm}

%\printglossary[type=\acronymtype]

\printindex

\end{document}