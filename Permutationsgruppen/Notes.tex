% !TeX spellcheck = en_US
\documentclass[ngerman,a4paper,order=firstname]{mathscript}
\usepackage{mathoperators}
\usepackage{chemfig}
\usepackage{chemformula}
%TODO fix names of enviroments into german? might look into the original.

% % % local commands
\DeclareMathOperator{\Ad}{Ad}				% Adjoint
\DeclareMathOperator{\PSL}{PSL} 			% projective linear group 
\newcommand{\with}{\text{ with }}
\newcommand{\nd}{\text{ and }}
\newcommand{\for}{\text{ for }}
\renewcommand{\rhd}{\triangleright}
\renewcommand{\lhd}{\triangleleft} 			% normal subgroups
\DeclareMathOperator{\Set}{Set}				% Category of sets
\newcommand{\cat}{\mathcal C}				% some category
\DeclareMathOperator{\Vect}{Vect}			% Category of vector spaces
\DeclareMathOperator{\Grp}{Grp}				% Category of groups
\DeclareMathOperator{\QVect}{QVect}			% Cat of quasi vector bundles
\DeclareMathOperator{\Mod}{Mod}				% Cat of moduls
\newcommand{\EE}{\mathscr E}				% some cat E
\DeclareMathOperator{\Ann}{Ann}				% annihilator
\DeclareMathOperator{\morph}{Morph}				% cat of morphs
\newcommand{\Circlearrowleft}{\mathbin{\rotatebox[origin=c]{180}{$\circlearrowright$}}}
\DeclareMathOperator{\Cl}{Cl}				% conjugation class of something.
\renewcommand{\phi}{\varphi}				% always varphi for phi
\DeclareMathOperator{\Iso}{Iso}				% set of isomorphisms
\DeclareMathOperator{\Lin}{\mathscr L}		% set of continuous linear maps from V to W
\newcommand{\hset}[3]{\mathrm{H}^{#1}(#2;#3)} % set of equivalent cocycles over X
\newcommand{\isoset}[3]{\Phi^{#1}_{#2}(#3)} % isomorphism class of - F vector bundles
\DeclareMathOperator{\Open}{Open}				% inclusion category for presheafs
\newcommand{\presheaf}{\mathcal S}			% presheaf
\newcommand{\equival}{\Leftrightarrow}
\DeclareMathOperator{\Orb}{Orb}
\newcommand{\sest}{SeSt.}
\renewcommand{\phi}{\varphi}
\newcommand{\gen}[1]{\langle #1 \rangle}
\newcommand{\st}[1]{#1^{\ast}}
\DeclareMathOperator{\Sub}{Sub}
\newcommand{\dcup}{\dot\cup}

% % % % color note stuff
%\newcommand{\marganote}[1]{\textcolor{gray}{#1}}
%\newcommand{\fehmnote}[1]{\textcolor{red}{#1}}
\newcommand{\scyllanote}[1]{\textcolor{blue}{#1}}

% % % % % underline stuff
%\newcommand{\ul}[1]{\underline{#1}}

% Nice looking emptyset
\let\oldemptyset\emptyset
\let\emptyset\varnothing

% remove matrix columns limit
\setcounter{MaxMatrixCols}{20}

% get this stupid arrows:
%\usepackage{mathabx,graphicx}  % ---> add to mathoperators
%\def\Circlearrowleft{\ensuremath{%
%		\rotatebox[origin=c]{180}{$\circlearrowleft$}}}
%\def\Circlearrowright{\ensuremath{%
%		\rotatebox[origin=c]{180}{$\circlearrowright$}}}
%\def\CircleArrowleft{\ensuremath{%
%		\reflectbox{\rotatebox[origin=c]{180}{$\circlearrowleft$}}}}
%\def\CircleArrowright{\ensuremath{%
%		\reflectbox{\rotatebox[origin=c]{180}{$\circlearrowright$}}}}
%\begin{document}
%	\Huge
%	$\circlearrowleft \circlearrowright $
%	
%	$\Circlearrowleft \Circlearrowright $
%	
%	$\CircleArrowleft \CircleArrowright $

% % % local packages
\usepackage{braids}

\newlist{remarkenum}{enumerate}{1}
\setlist[remarkenum]{label=(\alph*),ref=\theremark~(\alph*)}
\crefalias{remarkenumi}{remark}

\newlist{propenum}{enumerate}{1}
\setlist[propenum]{label=(\alph*),ref=\theproposition~(\alph*)}
\crefalias{propenumi}{proposition}

\newlist{expenum}{enumerate}{1}
\setlist[expenum]{label=(\alph*),ref=\theexample~(\alph*)}
\crefalias{expenumi}{example}

\newlist{lemmaenum}{enumerate}{1}
\setlist[lemmaenum]{label=(\alph*),ref=\thelemma~(\alph*)}
\crefalias{lemmaenumi}{lemma}

\newlist{defenum}{enumerate}{1}
\setlist[defenum]{label=(\roman*),ref=\thedefinition~(\roman*)}
\crefalias{defenumi}{definition}

\newlist{theoenum}{enumerate}{1}
\setlist[theoenum]{label=(\roman*),ref=\thedefinition~(\roman*)}
\crefalias{theoenumi}{theorem}

\title{\textbf{Permutationsgruppen WS 20/21}}
\author{Prof. Pöschel}

\begin{document}
\pagenumbering{roman}
\pagestyle{plain}

\maketitle

\hypertarget{tocpage}{}
\tableofcontents
\bookmark[dest=tocpage,level=1]{Inhaltsverzeichnis}

\pagebreak
\pagenumbering{arabic}
\pagestyle{fancy}

\section*{Preface}
This notes are loosly based on some notes of a two term Differential Geometry course and \cite{Lee}. They also contain structural notes, which i will mark with \scyllanote{Note} and those can also be wrong guesses, they will improve the more i think about this whole course, so take them with a grain of salt.

ScyllaHide, \today
\setcounter{section}{-1}
\section{Einführung}
\input{./TeX_files/sec0}
\section{Permutationen und Permutationsgruppen}
Permutationen können unterschiedlich definiert und dargestellt werden:
\begin{enumerate}
	\item als Lineare Anordnung von Elementen einer Menge, z.B. $M = \set{a,b,c}$
	\begin{align*}
		\pi_1\colon abc &\qquad \pi_2 \colon acb\\
		\pi_3 \colon bac &\qquad \pi_4 \colon bca\\
		\pi_5 \colon cab &\qquad \pi_6\colon cba
	\end{align*}
	\item als bijektive Abbildung (in 2-Zeilen-Darstellung)
	\begin{align*}
		\pi_1 = \begin{pmatrix}
			a & b & c\\
			a & b & c
		\end{pmatrix},\;
	\pi_2 = \begin{pmatrix}
		a & b & c\\
		a & c & b
	\end{pmatrix}, \dots, \pi_6 = ...
	\end{align*}
	allgemein
	\begin{align*}
		\pi = \begin{pmatrix}
			a_1 & \dots &a_n\\
			\vdots & & \vdots\\
			a_{1i} & \dots & a_{1n}
		\end{pmatrix}
	\end{align*}
	bezeichnen. Also Abbildung $\pi \colon M \to M$ mit $a_k \mapsto a_{i_k}$, wobei $M = \set{a_1, \dots, a_n}$ Reihenfolge der Spalten spielt keine Rolle.
\end{enumerate}
\begin{definition}
	Eine \begriff{Permutation} auf Menge $M$ ist bejektiv Abbildung $f \colon M \to M$
	\begin{align*}
		S_m := S(M) := \text{Menge aller Permutationen auf }M
	\end{align*}
	Bezeichnung: für Bild $f(a)$ eines Elementes $a \in M$ $a^f$ also ist
	\begin{align*}
		f = \begin{pmatrix}
			a_1 & \dots & a_n\\
			a^f_1 &\dots & a^f_n
		\end{pmatrix}
	\end{align*}
\end{definition}
\begin{proposition}
	Für $\abs{M} = n$ gibt $n!$ viele Permutationen auf $M$.
	\begin{align*}
		\abs{S_M} = n!
	\end{align*}
\end{proposition}
\begin{proof}
	Selbststudium!
\end{proof}
\begin{definition}
	Der Graph einer Permutation
	\begin{itemize}
		\item $f\colon M \to M$
		\item $f^{\cdot} = \set{(a,b) \in M^2 \mid a^f = b}$ \begriff{Graph} von $f$, Paare $(a,b)$ als gerichtete Kanten von $a$ nach $b$ zeichnen.
		\item Graph $f^{\cdot}$ (genauer: $(M,f^{\cdot})$ hat Knotenpunktmenge $M$ und Knotenmenge $f^{\cdot}$
	\end{itemize}
\end{definition}
\begin{example}
	\begin{align*}
		f = \begin{pmatrix}
			1 & 2 & 3 & 4 & 5 & 6 & 7 & 8 & 9 & 10 & 11 & 12 & 13\\
			3 & 5 & 7 & 12 & 2 & 9 & 4 & 11 & 8 & 1 & 6 & 12 & 10
		\end{pmatrix}
	\end{align*}
	\chemfig{1*6(-3-7-4-13-10-)}\\
	\chemfig{6*4(-9-8-11-)}
	\[
		% https://tikzcd.yichuanshen.de/#N4Igdg9gJgpgziAXAbVABwnAlgFyxMJZABgBpiBdUkANwEMAbAVxiRACYQBfU9TXfIRQBGclVqMWbAKzdxMKAHN4RUADMAThAC2SMiBwQkokACMYYKEgDMxHuq27EJw3urnLSACwBOLhS4gA
		\begin{tikzcd}
			2 \arrow[r, bend left] & 5 \arrow[l, bend left=49]
		\end{tikzcd}
		% https://tikzcd.yichuanshen.de/#N4Igdg9gJgpgziAXAbVABwnAlgFyxMJZABgBpiBdUkANwEMAbAVxiRAEYAmEAXypBhQA5vCKgAZgCcIAWyRkQOCPOoMIENETLjGcGPwZ0ARjAYAFTLnyFEISViEALHLwo8gA
		\begin{tikzcd}
		12 \arrow[loop, distance=2em, in=305, out=235]
		\end{tikzcd}
	\]
Fakt Vorraussetzung $M$ endlich: Graph $f^{\cdot}$ einer Permutation $f$ ist ein Kreis (Zyklus) oder die Vereinigung  von paarweisen disjunkten Kreisen (Zyklen) (folgt aus Bijektivität) (Gilt nicht für unendliche Mengen $f \colon \Z \to \Z \mit x \mapsto x+1$)
\end{example}
Ab jetzt $M$ endlich.
\begin{definition}
	Die \begriff{Zyklendarstellung einer Permutation $f$} (entspricht ``lineares Aufschreiben von $f^{\cdot}$'')
\end{definition}
\begin{example}
	$f$ wie oben
	\begin{align*}
		(1,3,7,4,13, 10)(2,5)(6,9,8,11)(12)	
	\end{align*}
\end{example}
Falls $M$ fest, Zyklen der Menge 1 weglassen, das nennt man die \begriff{verkürtzte Zyklendarstellung}. \begriff{zyklische Permutation} := Permutation mit genau einem Zyklus in der verkürtzten Zyklendarstellung. Die identische Permutation $x \mapsto x$, Zyklendarstellung $(1)$.\\
Beachte: $(abc),(bca),(cab)$ bezeichnen die selbe Permutation. (nur Reihenfolge, nicht Anfangselement wichtig)
\begin{definition}
	\label{sec1:def:multiplication_permu}
	Die \begriff{Multipllikation} (Produkt) von Permutationen ist die Hintereinanderausführung von Abbildungen.
	\[
	% https://tikzcd.yichuanshen.de/#N4Igdg9gJgpgziAXAbVABwnAlgFyxMJZABgBpiBdUkANwEMAbAVxiRAFkQBfU9TXfIRQBGclVqMWbTjz7Y8BIgCYx1es1aIO3XiAzzBRMsPHqpWujrkDFI0ibWTNIOgD0AZlb38FQ5CocJDTY3YHcAcy5ucRgocPgiUHcAJwgAWyQyEBwIJFEQBjoAIxgGAAUfQy0GGHccEEdgrU9ZEBT0vOocpBUg8xBwr3aMxABmLtzEABYuuiwGNjS6NDhu1uGkGezJgFZZ+cXl1dyuCi4gA
	\begin{tikzcd}
		M \arrow[r, "f"]     & M \arrow[r, "g"]       & M      \\
		a \arrow[r, maps to, "f"] & a^f \arrow[r, maps to, "g"] & a^{fg}
	\end{tikzcd}
	\]
	Produkt $fg$ (oder $f;g$ oder $f\cdot g$, oder auch $g \circ g$) wird definiert durch 
	\begin{align*}
		a^{fg}:= (a^f)^g
	\end{align*} 
	ist wieder Permutation.
\end{definition}
\begin{example}
	\begin{align*}
		\begin{pmatrix}
			1 & 2 &3\\
			2 & 1 &3
		\end{pmatrix}
		\begin{pmatrix}
			1 & 2 & 3\\
			3 & 2 & 1
		\end{pmatrix}=
		\begin{pmatrix}
			1 & 2 & 3\\
			2 & 3 & 1
		\end{pmatrix}
	\end{align*}
	\begin{itemize}
		\item Zykeldarstellung: $(12)(3)\cdot (13)(2) = (123)$
		\item verkürtzt: $(12)\cdot(13) = (123)$ 
	\end{itemize}
\end{example}
Fakt: verkürtzte Zyklendarstellung $k$ Zyklen
\begin{align*}
	f = (--c_1--)(--c_2--)\dots (--c_3--)\\
	g = (--c_i--)
\end{align*}
$g$ also Permutation zyklisch.
Also haben wir
\begin{align*}
	f = g_1 \cdot g_2 \cdot \cdots \cdot g_k
\end{align*}
(Komposition)
\begin{proposition}
	Die Permutationen aus $S_M$ bilden mit der Multiplikation eine Gruppe. die volle symmetrische Gruppe vom Grad $\abs{M}$. (Einselement $(1)$ = identische Abbildung, inverse Abbildung $f^{-1}$ Zeilen in 2-Zeilendarstellung vertauschen. $(ab \dots xy)^{-1} = (yx\dots ba)$.
\end{proposition}
\begin{proof}
	\sest
\end{proof}
\begin{remark}
	Alle gruppentheoretische Begriffe sind auch für Permutationsgruppen definiert:Ordnung $\ord(f) = \min\set{m \mid f^m = e}$, Untergruppen, Normalteiler, konjugierte Elemente, usw.
\end{remark}
% 2nd lecture 3.11.2020
\begin{definition}
	Eine \begriff{Permutationsgruppe} $G$ von Grad $n$ ist eine Untergruppe der vollen symmetrischen Gruppe $S_M$ vom Grad $n$.\\
	Bezeichnung: $(G,M)$ oder auch falls $G$ Untergruppe ist, $G \le S_M$. Wobei meist
	\begin{align*}
		M &= \set{0,1,\dots,n-1} =: n \implies S_n\\
		M &= \set{1,2,\dots, n} =: \uline{n} \implies S_{\uline{n}}
	\end{align*}
\end{definition}
\begin{remark}
	Weitere Schreibweisen für $U,V \subseteqq S_M$, dabei
	\begin{align*}
		UV &= \set{uv \mid u\in U, v \in V}
		\intertext{$a\in M, B\subseteqq M, g \in S_M$}
		a^u &= \set{a^u \mid u \in U}\\
		B^g &= \set{b^g \mid b \in B}\\
		B^u & = \set{b^u \mid b \in B, u \in U}
	\end{align*}
\end{remark}
\begin{proposition}[Gruppenkriterium]
	Sei $M$ endlich. Dann ist $U \subseteqq S_M$ Gruppe genau dann, wenn $U\cdot U \subseteqq U$.
\end{proposition}
\begin{proof}
	\sest
\end{proof}
\begin{example}
	Symmetrieabbildung des Rechtecks in der Ebene
%	\begin{center}
%		\tikzset{every picture/.style={line width=0.75pt}} %set default line width to 0.75pt        
%		
%		\begin{tikzpicture}[x=0.75pt,y=0.75pt,yscale=-1,xscale=1]
%			%uncomment if require: \path (0,300); %set diagram left start at 0, and has height of 300
%			
%			%Shape: Rectangle [id:dp8018685777285718] 
%			\draw   (100,110) -- (159.83,110) -- (159.83,169.67) -- (100,169.67) -- cycle ;
%			%Straight Lines [id:da9350361269288623] 
%			\draw  [dash pattern={on 0.84pt off 2.51pt}]  (79.83,139.67) -- (179.83,139.67) ;
%			%Straight Lines [id:da8951381985636013] 
%			\draw  [dash pattern={on 0.84pt off 2.51pt}]  (129.5,90) -- (129.83,190.67) ;
%			
%			% Text Node
%			\draw (161,95) node [anchor=north west][inner sep=0.75pt]  [font=\footnotesize] [align=left] {2};
%			% Text Node
%			\draw (161.83,172.67) node [anchor=north west][inner sep=0.75pt]  [font=\footnotesize] [align=left] {3};
%			% Text Node
%			\draw (90,97) node [anchor=north west][inner sep=0.75pt]  [font=\footnotesize] [align=left] {1};
%			% Text Node
%			\draw (88,173) node [anchor=north west][inner sep=0.75pt]  [font=\footnotesize] [align=left] {4};
%			% Text Node
%			\draw (124,70) node [anchor=north west][inner sep=0.75pt]  [color={rgb, 255:red, 245; green, 166; blue, 35 }  ,opacity=1 ] [align=left] {II};
%			% Text Node
%			\draw (182,131) node [anchor=north west][inner sep=0.75pt]  [color={rgb, 255:red, 245; green, 166; blue, 35 }  ,opacity=1 ] [align=left] {I};
%			
%			
%		\end{tikzpicture}
%		
%	\end{center}
können durch Permutation der Eckpunkte beschrieben werden. Also
\begin{center}
	\begin{tabular}{c|c}
		Identitätsabbildung: & $(1) =: e$\\
		\hline
		Drehung 180°: & $(13)(24) =: g_1$\\
		Spiegelung an I: & $(14)(23) =: g_2$\\
		Spiegelung an II: & $(12)(34) =: g_3$  
	\end{tabular}
\end{center}
Damit ist $G = \set{e,g_1,g_2,g_3}$ Permutationsgruppe und $G \cong$ Symmetriegruppe des Rechtecks in der Ebene.
\begin{center}
	\begin{tabular}{c|cccc}
		$\cdot$ & $e$ 	& $g_1$ & $g_2$ & $g_3$\\
		\hline
		$e$   	& $e$   & $g_1$ & $g_2$ & $g_3$\\
		$g_1$ 	& $g_1$ & $e$   & $g_3$ & $g_2$\\
		$g_2$ 	& $g_2$ & $g_3$ & $e$   & $g_1$\\
		$g_3$ 	& $g_3$ & $g_2$ & $g_1$ & $e$\\
	\end{tabular}
\end{center}
das ist die \person{Klein}sche Vierergruppe, die isomorph zu $\Z_2 \times \Z_2$ ist.
\end{example}
\begin{definition}
	Sei $(G,M)$ Permutationsgruppe, $a\in M$, dann definiere
	\begin{enumerate}
		\item Den \begriff{Stabilisator} von $a$
		\begin{align*}
			G_a := \set{g \in G \mid a^g = a}
		\end{align*}
		Allgemeiner haben wir 
		\begin{align*}
			G_{a_1, a_2, \dots, a_m} := \bigcap_{i=1}^m G_{a_i}.
		\end{align*}\label{sec1:exm:Orbit}
		\item Die \begriff{Bahn} (1-Bahn, Orbit)
		\begin{align*}
			a^G := \set{a^g \mid g \in G}.
		\end{align*}
		Also ist der 1-$\Orb(G,M):=$ Menge aller 1-Bahnen (Andere Bezeichnung: $G\Vert M$).
		\item $B \subseteq M$ \begriff{invariante Menge} (bezüglich $G$) $:\equival B^G \subseteqq B$ 
		\item $G$ \begriff{transitiv} $\equival \exists a \in M \colon a^G = M$
		\begin{*remark}
			Äquivalent dazu sind:
			\begin{align*}
				\forall a \in M \colon a^G = M\\
				\abs{1-\Orb(G,M)} = 1
			\end{align*}
		\end{*remark}
	\end{enumerate}
\end{definition}
\begin{lemma}
	Sei $G \le S_M, a \in M$. Es gilt:
	\begin{lemmaenum}
		\item $G_a$ ist Untergruppe von $G$. \label{lem_1.11_1}
		\item $G_{a^g} = g^{-1}(G_a)g$ $(g \in G)$ \label{lem_1.11_2}
		\item Durch $a \sim b \equival a^G = b^G$ ist eine Äquivalenzrelation gegeben und es gilt:
		\begin{align*}
			1-\Orb(G,M) = M/\sim
		\end{align*}
		Die menge aller 1-Bahnen bildet eine Zerlegung von $M$ (zwei Bahnen sind gleich oder disjunkt).\\
		Beachte:
		\begin{align*}
			b \in a^G \equival a \in b^G \quad \text{ Sest!}
		\end{align*}
		\label{lem_1.11_3}
		\item Jede invariante Menge $B \subseteqq M$ ist Vereinigung von 1-Bahnen
		\begin{align*}
			B = B^G = \bigcup_{b \in B} b^G
		\end{align*}
		\label{lem_1.11_4}
	\end{lemmaenum}
\end{lemma}
\begin{proof}
	\cref{lem_1.11_1} - \cref{lem_1.11_3} SeSt, \cref{lem_1.11_4} klar nach Definition!
\end{proof}
\begin{remark}
	Repräsentatensystem einer Zerlegung $1-\Orb(G,M)$ heisst \begriff{Transversale}.
\end{remark}
\subsection*{Wiederholung Algebra}
\begin{proposition}[Satz von \person{Lagrange}]\label{prop:lagrange}
	Die Ordnung $\abs{U}$ jeder Untergruppe einer endlichen Gruppe $G$ ist Teiler der Gruppenordnung. Es gilt
	\begin{align*}
		\abs{G} = [G\colon U]\cdot \abs{U}
	\end{align*}
	(wobei $[\cdot\colon \cdot]$ der Index ist).
\end{proposition}
\begin{proof}
	Index $[G\colon U] = \abs{G/U}$ = Anzahl $k$ der (rechts-)Nebenklassen $Ug$ in der Nebenklassenzerlegung.
	\begin{align*}
		G = U g_1 \cup U g_2 \cup \dots U g_k
	\end{align*}
	Dabei ist die Nebenklasse durch $Ug := \set{u\cdot g \mid u \in U}$ und $G/U := \set{Ug \mid g \in G}$ und wegen $\abs{U} = \abs{Ug}$ folgt \cref{prop:lagrange}.
\end{proof}
\begin{lemma}\label{lem:stab}
	Sei $a \in M, G \le S_M$.
	\begin{align*}
		\begin{cases}
			a^G &\to G/G_a\\
			a^g &\mapsto G_a g
		\end{cases}
	\end{align*}
	ist eine bijektive Abbildung zwischen Elementen der von $a$ erzeugten Bahn un den Nebenklassen nach dem Stabilisator $G_a$ gegeben.
	\scyllanote{Insbesondere gilt:
	\begin{align*}
		\abs{a^G} = [G\colon G_a] = \abs{G/G_a}
	\end{align*}
	}
\end{lemma}
\begin{proof}
	\begin{align*}
		a^g = a^{g'} &\overset{\scyllanote{\equival}}{\implies} a= a^{g'g^{-1}}\overset{\scyllanote{\equival}}{\implies} g'g^{-1} \in G_a\\
		&\overset{\scyllanote{\equival}}{\implies} g' \in G_a g\overset{\scyllanote{\equival}}{\implies} G_a g' = G_a g
	\end{align*}
	(letzte $\implies$ benutzt Nebenklassen gleich oder disjunkt)
	\begin{itemize}
		\item $\implies$ zeigt, dass $a^g \mapsto G_a g$ wohldefiniert ist
		\item $\scyllanote{\Leftarrow}$ zeigt, dass $a^g \mapsto G_a g$ injektiv ist.
		\item Surjektivität ist klar, da $g$ beliebig gewählt werden kann.
	\end{itemize}
\end{proof}
Nun formulieren wie eine Folgerung, die \cref{prop:lagrange} und \cref{lem:stab}.
\begin{conclusion}[Permutationsgruppentheoretische Umformulierung des \cref{prop:lagrange}]\label{folg:permu_langrange}
	Für $a \in M$, $G \le S_M$ gilt:
	\begin{align*}
		\abs{G} = \abs{G_a} \cdot \abs{a^G}
	\end{align*}
\end{conclusion}
\begin{proof}
	\begin{align*}
		\abs{G} \overset{\cref*{prop:lagrange}}&{=} [G\colon U]\cdot \abs{U} = [G \colon G_a]\cdot [G_a]\\
		&= \abs{G/G_a}\cdot \abs{G_a} \overset{\cref*{lem:stab}}{=} \abs{a^G}\cdot \abs{G_a}
	\end{align*}
\end{proof}
\begin{example}
	Sei $G:= S_{\ul{4}}$, $1^G = \set{1,2,3,4}$ und $M = \ul{4} = \set{1,2,3,4}$, dann
	\begin{align*}
		\abs{G_1} \overset{\cref*{folg:permu_langrange}}&{=} \abs{G} / \abs{1^G} = 4! / 4 = 3! = 6 
	\end{align*}
	``Raten'' der Permutationen aus $G_1$ führt zu
	\begin{align*}
		G_1 = \set{(1), (23), (24), (34), (234), (243)}
	\end{align*}
	(mehr als 6 gibt es nicht!)\\
	Iteration führt zu
	\begin{align*}
		\abs{G_{1,2}} &= \abs{G_1}:\abs{2^{G_1}} = 6:3 = 2\\
		G_{1,2} &= \set{(1), (34)}\\
		\abs{G_{1,2,3}} &= \abs{G_{1,2}}:\abs{3^{G_{1,2}}} = 2:2 = 1\\
		G_{1,2,3} &= \set{(1)}
	\end{align*}
\end{example}
\begin{definition}\label{sec1:def:similar}
	Zwei Permutationsgruppen (bzw. Wirkungen, siehe 2.2) %TODO add reference later!
	$(G,M)$ und $(H,N)$ heißen \begriff{ähnlich}, wenn eine bijektive Abbildung  
	\begin{align*}
		f\colon M \to N
	\end{align*}
	und ein Gruppenisomorphismus
	\begin{align*}
		\phi\colon G \to H
	\end{align*} 
	existieren, so dass gilt:
	\begin{align*}
		\forall a \in M, \forall g\in G\quad f(a^g) = f(a)^{\phi(g)}
	\end{align*}
	d.h. das Diagramm
	\[
	% https://tikzcd.yichuanshen.de/#N4Igdg9gJgpgziAXAbVABwnAlgFyxMJZABgBpiBdUkANwEMAbAVxiRAFkQBfU9TXfIRQBGclVqMWbAHLdeIDNjwEio4ePrNWiELJ58lgomXXVNUnZy7iYUAObwioAGYAnCAFskZEDghJhfRA3TyQAZmo-JAAmM0ltYLkXdy9EUV9-RFiJLTYAHTy0AAssAAo7AEok4JTvSMyInIsQO2qQ1J8otLjcnWduCi4gA
	\begin{tikzcd}
		M \arrow[r] \arrow[d, "g"] \arrow[r, "f"] & N \arrow[d, "\phi(g)"] \\
		M \arrow[r, "f"]                          & N                     
	\end{tikzcd}
	\]
	ist kommutativ.
\end{definition}
\begin{*remark}
	Durch $f$ und $G$ ist $\phi$ und $H$ vollständig festgelegt (folgt aus Diagramm)
	\begin{align*}
		y \in N \quad y^{\phi(g)} = g^{f^{-1}gf}\\
		H = \set{\phi(g) \mid g \in G} = \set{f^{-1}gf \mid g \in G}
	\end{align*}
\end{*remark}
Beachte: Ähnlichkeit impliziert Isomorphie, Äquivalenz gilt im Allgemeinen nicht!
\begin{example}
	\begin{enumerate}
		\item $S_M$ ähnlich zu $S_N \implies \abs{M} = \abs{N}$
		\item $(G,M) = (\set{e, (12)}, \set{(1,2)})$ ähnlich zu $(\set{e, (\alpha, \beta)}, \set{\alpha, \beta}) = (H,N)$ aber nicht ähnlich zu $(\set{e, (1,2)}, \set{1,2,3,4}) = (G',M')$, obwohl $G \cong G'$ (als Gruppen).
	\end{enumerate}
\end{example}
\begin{definition}
	Ähnlichkeit und Konjugiertheit von Permutationen
	\begin{enumerate}
		\item zwei Permutationen $g_1, g_2 \in S_M$ heißen \begriff{ähnlich}, wenn in ihren Zyklendarstellungen gleich viele Zyklen, gleicher Länge vorkommen, z.B.
		\begin{align*}
			g_1 =& (1)(2)(345)(67)(89)\\
			g_2 =& (3)(7)(149)(28)(56)\\
				 & \scyllanote{(\cdot)(\cdot)(\cdot\cdot\cdot)(\cdot\cdot)(\cdot\cdot)}
		\end{align*}
		\item Sei $G \le S_M$, $g_2 \in S_M$ heißt \begriff{konjugiert} zu $g_1 \in G$, wenn ein $f \in G \scyllanote{\neq S_M}$ existiert, so dass $g_2 = f^{-1}g_1 f$.\\
		Sprechweise: $g_1$ und $g_2$ sind konjugiert.
	\end{enumerate}
\end{definition}
\begin{lemma} % L. 1.19
	\begin{enumerate}
		\item Konjugiertheit und Ähnlichkeit sind Äquivalenzrelation (in $S_{\ul{n}}$).
		\item Aus der Zyklendarstellung
		\begin{align*}
			g = (a_1, a_2, \dots)(b_1, b_2, \dots)(\dots)\dots
		\end{align*}
		erhält man die Zyklendarstellung von $f^{-1}gf$ (für $f \in S_{\ul{n}}$), wenn $f$ auf jedes Element im Zyklus angewendet wird
		\begin{align*}
			f^{-1}gf = (a_1^f, a_2^f, \dots)(b_1^f, b_2^f, \dots)(\dots)\dots
		\end{align*}
		\item $g_1$ und $g_2$ konjugiert $\implies$ $g_1, g_2$ ähnlich (\scyllanote{$\Leftarrow$} im Allgemeinen nicht!), aber, wenn $g_1 \vee g_2$ konjugiert in $S_{\ul{n}}$ $\equival$ $g_1,g_2$ ähnlich
		\item $g_1, g_2 \in S_{\ul{n}}$ ähnlich $\equival$ die erzeugte zyklische Untergruppe $(\gen{g_1},M)$ und $(\gen{g_2},M)$ sind ähnlich im Sinne von der Definition der Untergruppe (\cref{sec1:def:similar}).
	\end{enumerate}
\end{lemma}
\section{Gruppenwirkungen und Darstellungen (Satz von \person{Cayley})}
% !TeX spellcheck = de_DE
%10.11.2020
\begin{definition}[Permutationsdarstellung]
	\label{sec2:def:permudarstellung}
	\begin{defenum}
		\item Ein Homomorphismus
		\begin{align*}
			\psi\colon G \to S_M
		\end{align*}
		einer (abstrakten) Gruppe $G$ in eine symmetrische Gruppe $S_M$ heisst \begriff{Permutationsdarstellung} von $G$. (vom Grad $\abs{M}$)
		\item $\psi$ \begriff{true}, falls $\psi$ injektiv.
	\end{defenum}
\end{definition}
\begin{*remark}
	$\psi$ ist treu $\equival$ $\ker \psi = \set{g \mid \psi(g) = e} = \set{e}$, mithilfe des Homomorphiesatz folgt $G \cong \Im \psi(G) = \psi[G] \le S_M$. ($G$ ist ``praktisch'' Permutationsgruppe.)
\end{*remark}
\begin{definition}
	Sei $G = \gen{G, \cdot, ^{-1}, e}$ Gruppe, $M$ Menge. Eine Abbildung
	\begin{align*}
		\phi \colon \begin{cases}
			M \times G &\to M\\
			(x,g) &\mapsto xg = \phi(x,g)
		\end{cases}
	\end{align*}
	heisst \begriff{Gruppenwirkung}(alt: Gruppenoperation, eng. group action) von $G$ auf Menge $M$, falls folgendes gilt:
	\begin{enumerate}
		\item $\phi(x,e) =xe_G = x \quad \forall x \in M$
		\item 
		\begin{align*}
			(xg)g' &= x(g\cdot_G g')\\
			\phi(\phi(x,g),g') &= \phi(x,g\cdot_G g')
		\end{align*}
	\end{enumerate}
	Sprechweise: $G$ \ul{wirkt} (operiert, eng. acts) auf $M$\\
	Schreibweise: $(G,M)$
\end{definition}
\begin{*remark}
	Jede Permutationsgruppe $G \le S_M$ operiert auf natürliche Weise auf $M$
	\begin{align*}
		xg = \phi(x,g) = x^g
	\end{align*}
	(oft Schreibweise $x^g$ statt $\phi(x,g) \wedge xg$)
\end{*remark}
\begin{proposition}\label{sec2:prop:groupaction}
	Jeder Gruppenwirkung
	\begin{align*}
		\phi\colon M\times G \to M
	\end{align*}
	entspricht in ein-eindeutiger Weise einer Permutationsdarstellung.\\
	$\psi\colon G \to S_M$ und umgekehrt, und zwar gemäß
	\begin{align*}
		x^{\psi(g)} &= xg & (= \phi(x,g)\\
		(x\in M, g\in G) &:= & \text{falls $\phi$ gegeben}\\
		& := &\text{falls $\psi$ gegeben}
	\end{align*}
\end{proposition}
\begin{proof}
	\sest. (Hinweis: Es muss gezeigt werden, dass $\psi$ ein Gruppenhomomorphismus ist.)
\end{proof}
\begin{lemma}
	\begin{lemmaenum}\label{sec2:lem:representation_permugroups}
		\item Ist $G$ (abstrakte) Gruppe, so ist durch
		\begin{align*}
			h \in G\quad h^{\st{g}}
		\end{align*}
		(rechts-multiplikation mit $g$) für jedes $g \in G$ eine Permutation: $\st{g}\in S_{G}$ gegeben.\label{sec2:lem:reppermu:rimu}
		\item $\psi\colon G \to S_G$ mit $g \mapsto \st{g}$ ist Permutationsdarstellung\label{sec:lem:reppermu:rep}
		\item $\phi\colon (G\times G) \to G$ mit $(h,g) \mapsto hg$ (Produkt in Gruppe $G$) zugehörige Gruppenwirkung\label{sec2:lem:reppermu:groupact}
		\item $\phi$ ist treu (und heißt \begriff{rechtsreguläre Darstellung} von $G$) \label{sec2:lem:actionfaithful}
	\end{lemmaenum}
\end{lemma}
Folgerung aus \cref{sec2:lem:actionfaithful} (vergleiche Bemerkung zu \cref{sec2:def:permudarstellung}) ist
\begin{conclusion}[Satz von \person{Cayley}]
	Für eine beliebige Gruppe $G$ ist
	\begin{align*}
		\st{G} = \set{\st{g} \mid g \in G} \subset S_G
	\end{align*}
	eine zu $G$ isomorphe (da treu) Permutationsgruppe $(\st{G},G)$ heißt \begriff{rechtsreguläre Darstellung} von $G$.
\end{conclusion}
\begin{proof}[\cref{lem:representation_permugroups}]
	\begin{itemize}
		\item \cref{sec2:lem:reppermu:rimu} und \cref{sec:lem:reppermu:rep} folgen wegen \cref{sec2:prop:groupaction} aus 3)
		\item zu zeigen \cref{sec2:prop:groupaction} für $\phi$
		\begin{enumerate}
			\item $\phi(h,g) = he = h$
			\item $(hg)g^{-1} = h(gg^{-1})$ (assoziativ Gruppen der Gruppenmultiplikation)
		\end{enumerate}
		\item noch zu zeigen \cref{sec2:lem:reppermu:groupact}: Sei $\st{g}_1 = \st{g}_2$ (gilt unter $\psi$). Dann $h^{\st{g}_1} = h^{\st{g}_2}$ impliziert $\xRightarrow{h^{-1}} g_1 = g_2$. (Da $h^{-1} \cdot h = e$ gilt.)
	\end{itemize}
\end{proof}
\begin{example}
	Sei $G = S_3 = \set{g_1, g_2, g_3, g_4, g_5, g_6}$ mit $M = \set{1,2,3}$
	\begin{align*}
		\begin{matrix}
			g_1 = e = (1) & g_2 = (12) & g_3 = (13)\\
			g_4 = (23) & g_5 = (123)  & g_6 = (132)
		\end{matrix}
	\end{align*}
	Multiplikationstafel (Cayley table)
	%TODO fixxxx
%	\[
%	\begin{center}
%		\begin{tabular}{c|cc|c|ccc}
%			$\cdot$ & $g_1$ & $g_2$ & $g_3$ & $g_4$ & $g_5$ & $g_6$\\
%			\hline
%			$g_1$   & & & $g_3$ & & \\
%			$g_2$ 	& & & $g_5$ & & \\
%			$g_3$ 	& & & $g_1$ & & \\
%			$g_4$ 	& & & $g_6$ & & \\
%			$g_5$ 	& & & $g_2$ & & \\
%			$g_6$ 	& & & $g_4$ & & \\
%		\end{tabular}
%	\end{center}
%	\]
	then we have: $\st{g}_i g \to gg_i$ wird durch die 3. Spalte des Cayleytafel beschrieben, d.h.
	\begin{align*}
		\st{g}_3 = \begin{pmatrix}
			g_1 & g_2 & g_3 & g_4 & g_5 & g_6\\
			g_3 & g_5 & g_1 & g_6 & g_2 & g_4
		\end{pmatrix} = (g_1 g_3)(g_2 g_5)(g_3 g_1)
	\end{align*}
	Zyklenschreibweise.
\end{example}
\begin{*remark}
	\begin{enumerate}
		\item $G$ wirkt $S_{\ul{3}}$ auf $M = \set{1,2,3}$
		\item $\st{G} = \st{S}_{\uline{3}}$ wirkt auf der Menge $S_{\uline{3}} = \set{g_1, \dots, g_6}$ d.h. ist Untergruppe der $S_G$, SeSt kein $\st{g} \neq e$ hat eine Fixpunkt
		\item Jedes $\st{g}$ zerfällt in ein Produkt von Zyklen gleicher Menge $\ord(g)$, vgl. vorheriges Beispiel: $\ord(g_2) = 2$
		\item $\st{G}$ hat Grad $\abs{G}$.
		\item $\st{G}$ ist transitiv (d.h. gibt nur eine Bahn $e^{\st{G}} = G$)
		\item Die EGS 2. und 5. charakterisieren die Regularität von $\st{G}$ (vgl 5.4) 
	\end{enumerate}
\end{*remark}
\begin{example}[Weitere Beispiele von Gruppenwirkungen einer (abstrakten) Gruppe $G$]
	\begin{enumerate}
		\item Wirkung durch Konjugation (siehe Geometrie Kurs)
		\begin{align*}
			\phi \colon \begin{cases}
				G \times G &\to G\\
				(h,g) &\mapsto g^{-1}hg
			\end{cases}
		\end{align*}
		zugehörige Permutationsdarstellung:
		\begin{align*}
			\psi \colon \begin{cases}
				G &\to S_G\\
				g &\to \psi(g)
			\end{cases}\\
		h^g = h^{\psi(g)} = g^{-1}hg
		\end{align*}
	\item Wirkung auf Untergruppen $U \subseteq G$ ($\Sub(G)$ Menge der Untergruppen von $G$)
		\begin{align*}
			\psi \colon \begin{cases}
				\Sub(G) \times G &\to \Sub(G)\\
				(U,g) &\to g^{-1}Ug
			\end{cases}
		\end{align*}
	zugehörige Permutationsdarstellung
	\begin{align*}
		\phi \colon G \to S_{\Sub(G)}
	\end{align*}
	\item Wirkung auf rechte-Cosets $G/U$
	\begin{align*}
		U \subseteq G, G/U = \set{Uh \mid h \in G}\\
		\phi \colon \begin{cases}
			G/U \times G &\to G/U\\
			(Uh,g) &\mapsto Uhg
		\end{cases}\\
	\phi \colon G \to S_{G/U}
	\end{align*}
	\end{enumerate}
\end{example}
\begin{example}[Wirkungen von Permutationsgruppen $(G,M)$ auf andere Mengen]\label{sec2:exm:actions}
	\begin{expenum}
		\item induzierte Wirkung $(G,\powerset(M))$ auf Potenzmenge $\powerset(M)$
		\begin{align*}
			\powerset(M) \times G &\to \powerset(M)\\
			(B,g) &\mapsto B^g = \set{h^g \mid h \in B} 
		\end{align*}\label{sec2:exm:action_a}
		\item Einschränkung: induzierte Wirkung $(G,\powerset_m(M))$ auf $m$-elementige Teilmenge
		\begin{align*}
			\phi\colon \begin{cases}
				\powerset_m(M) \times G &\to \powerset_m(M)\\
				(B,g) &\mapsto B^g \quad \abs{B} = m
			\end{cases}
		\end{align*}
		Bezeichnung dieser Wirkung auch $(G^{m}, \powerset_m(M))$ \label{sec2:exm:action_b}
		\item induzierte Wirkung von $(G,M^m)$ auf $m$-Tupel d.h. auf $M^m$
		\begin{align*}
			\phi \colon \begin{cases}
				M^m \times G &\to M^m\\
				(a_1, \dots, a_m)^g &\mapsto (a_1^g, \dots, a_m^g)
			\end{cases}
		\end{align*}
		Bezeichnung: $(G^{[m]}, M^m)$. \label{sec2:exm:action_c}
	\end{expenum}
\end{example}
\section{Erzeugendensysteme und SIMS-Ketten}
Problem: Beschreibung von Permutationsgruppen, Aufzählung aller Elemente ist nur selben möglich ($S_{\uline{100}} = 100 \implies 10^{100} \dots 10^{200}$)\\
Aufzählung: Beschreibung als Automorphismusgruppen (siehe Kapitel 4 und 5) oder durch EZS.\\
Wiederholung:
\begin{definition}
	$U \subseteq G$ heißt \begriff{Erzeugendensystem} einer Gruppe $G :\equival$ jedes $g \in G$ ist als endliches Produkt $u_1, \dots, u_m$ mit $u_i \in U$ oder $u^{-1}_i \in U$ darstellbar.\\
	Bezeichunung: $G = \langle u_i \rangle_G$.
\end{definition}
Probleme:
\begin{itemize}
	\item[(P1)] Entscheide $g \in \gen{U}$ für $g \in S_n, U \subseteq S_n$?
	\item[(P2)] Bestimme Bahnen von $\gen{U}$, spezielle Bahnen $a^{\gen{U}}$ für spezielle $a \in G$.
	\item[(P3)] Beschreibung der Untergruppen von $\gen{U}$, benutze Methode von \begriff{SIMS} für große $G$. Man benutzt Menge $T_i$ für $i =\set{1,\dots,r}$, sodass
	\begin{align*}
		G = T_r \cdot T_{r-1} \cdots T_1
	\end{align*}
	und Darstellung
	\begin{align*}
		g = t_r \cdot t_{r-1} \cdot t_{r-2}\cdots t_1
	\end{align*}
	ist Eindeuting.
\end{itemize}
Damit wäre die Speicherformel: $\sum_{i=1}^r \abs{T_i}$
\begin{*example}
	$G = S_{\uline{n}}$ impliziert $\abs{G} = n!$ oder $\sum \abs{T_i} \le \frac{n(n+1)}{2}$ möglich, also ist der Speicherbedarf $\sim n^2$.
\end{*example}
\begin{definition}[\person{Sims}-Kette, \person{Sims}-Basis, Transversale]
	\label{sec3:def:sims_stuff}
	Die Sims-Kette einer Permutationsgruppe $G \subseteq S_M$, $M = \set{a_1, \dots, a_n}$ speziell $M = \uline{n} = \set{1, \dots, n}$\\
	für punktweise Stabilisatoren:
	\begin{align*}
		\begin{matrix}
			U_1 = G_{n_1} & U_2 = G_{n_1,n_2} & \dots & U_i = G_{n_1,\dots,n_i} &
			U_n = G_{n_1,\dots,n_n} = \set{e}
		\end{matrix}
	\end{align*}
	Also haben wir
	\begin{align*}
		\set{e} = U_1 \subseteq U_2 \subseteq \cdots \subseteq U_i \subseteq \cdots \subseteq U = G
	\end{align*}
	Sei $r:= \min\set{i \mid U_i = \set{e}}$ (hängt von der Reihenfolge der Elemente $n_i$ ab). Die Menge der $\set{a_1, \dots, a_r}$ genauer $(a_1,\dots, a_r)$ heißt \begriff{SIMS-Basis} von $G$ und
	\begin{align*}
		\set{e} = U_r \lneqq U_{r-1} \leqq \cdots \leqq U_1 \leqq U_0 = 0
	\end{align*}
	ist die \begriff{SIMS-Kette} von $G$ der Länge $r$ (zur Basis $(a_1, \dots, a_r)$). Für
	\begin{align*}
		U_{i-1}/U_i = U_i g_{i_1} \dcup U_i g_{i_2} \dcup U_i g \dcup \dots \dcup U_i g_{i_{n_i}}\quad \text{meist }g_{i_1} = e
	\end{align*}
	wird Repräsentatensystem (\begriff{Transversale}) $T_i := \set{g_{i_1}, \dots, g_{i_{n_i}}}$ gewählt $(i = 1, \dots, r)$.\\
	Beachte: 
	\begin{align*}
		U_{r-1}/U_r \cong U_{r-1}\text{,also } T_r = U_{r-1}
	\end{align*}
\end{definition}
Bei Umnummerierung der Elemente entstehen möglicherweise kürzere Base. (Fixpunkte in Basis weggelassen)
\begin{proposition}
	Seien $G,T_i$ wie in \cref{sec3:def:sims_stuff}. Dann gilt
	\begin{defenum}
		\item Jede Permutation $g \in G$ lässt sich eindeutig in der Form
		\begin{align*}
			g = h_r h_{r-1} \cdots h_1 \mit h_i \in T_i\;(i \in \set{1, \dots, r})
		\end{align*}
		darstellen. Insbesondere gilt dann
		\begin{align*}
			G = T_r T_{r-1} \cdot \dots \cdot T_1 \nd \abs{G} = \prod_{i=1}^r n_i.
		\end{align*} \label{sec3:prop3:unique_simsform}
		\item Jede Permutation $g \in G$ ist eindeutig durch die Bilder der Basis festgelegt, d.h. durch $(a_1^g, \dots, a_r^g)$.\label{sec3:def:unique_image}
	\end{defenum}
\end{proposition}
\begin{*remark}
	\cref{sec3:prop3:unique_simsform} impliziert $T_1 \cup \dots \cup T_r$ ist ein (spezielles) Erzeugendensystem für $G$.
\end{*remark}
\begin{proof}
	\begin{itemize}
		\item zu \cref{sec3:prop3:unique_simsform}:
		\begin{align*}
			g \in G \implies \exists! h_1 \in T_1 \colon g \in U_1 h_1\\
			\implies gh^{-1}_1 \in U_1 \implies \exists! h_2 \in T_2 \colon gh^{-1}_1 \in U_2 h_2\\
			\implies gh^{-1}_1 h_2^{-1} \in U_2 \implies \exists! h_3 \dots\\
			\implies gh^{-1}_1 h_2^{-1} \dots h_r^{-1} \in U_r = \set{e}\\
			\implies g = h_r h_{r-1}\cdot \dots \cdot h_2\cdot h_1
		\end{align*}
		Eindeutigkeit der Darstellung folgt aus der Eindeutigkeit der Repäsentaten (der Nebenklassen).
		\item zu \cref{sec3:def:unique_image}:
		\begin{align*}
			(a_1^g, \dots, a_r^g) = (a_1^{g'}, \dots, a_r^{g'}) \implies a_i^{gg^{-1}} = a_i \quad \text{ Fixpunkte}
			\intertext{d.h.}
			gg^{-1} \in G_{a_1 a_2 \dots a_r} = \set{e} \implies g = g'
		\end{align*}
	\end{itemize}
\end{proof}
\begin{example}
	Sei $G = S_{\ul{4}}, M = \set{a_1, a_2, a_3, a_4} = \set{1,2,3,4}$, $G_1 \cong S_3$, $G_{1,2} \cong S_2$ und $G_{1,2,3} = \set{e}$, dann muss man etwas rechnen und bekommt
	\begin{align*}
		\begin{matrix}
			T_1 = \set{e,g_1, g_1^2, g_1^3} & \text{für } g_1=(1234)\\
			T_2 = \set{e,g_2, g_2^2} & \text{für } g_2=(234)\\
			T_1 = \set{e,g_3} & \text{für } g_3=(34)\\
		\end{matrix}
	\end{align*}
	Dann folgt mit \cref{sec3:prop3:unique_simsform}: Jedes $g \in S_{\ul{4}}$ ist eindeutig in der Form
	\begin{align*}
		g=g_3^{\alpha_3}g_2^{\alpha_2}g_1{\alpha_1}
	\end{align*}
	wobei $\alpha_3 \in \set{0,1}$, $\alpha_2 \in \set{0,1,2}$, $\alpha_1 = \set{0,1,2,3}$ und $g_0 = e$.
\end{example}
\begin{*remark}
	Speicheraufwand (in Bit):
	\begin{align*}
		\abs{T_1} = 4 &\implies 2 \text{ Bit}\\
		\abs{T_2} = 3 &\implies 2 \text{ Bit}\\
		\abs{T_3} = 2 &\implies 1 \text{ Bit}\\
				      &\implies 5 \text{ Bit}
	\end{align*}
	ist optimal, da wir $2^4 (16)\lneqq 4! (24) \le 2^5 (32)$ haben.
\end{*remark}
\begin{conclusion}[Test $g \in G$, vergleiche Probleme (P1) vom Anfang des Kapitels]
	Für $G \le S_M$ seinen eine SIMS-Basis $(a_1, \dots, a_r) \nd T_1, \dots, T_r$ bekannt (vergleiche \cref{sec3:def:sims_stuff}), $g \in S_M$ gegeben.\\
	\newcommand{\ja}{\text{ ja}} \newcommand{\nein}{\text{nein}}
	Algorithmus zum Testen, ob $g \in G$.
	\begin{align*}
		\begin{matrix}
			\exists h_1 \in T_1 \colon a_1^{gh_1^{-1}} = a_1? & \xrightarrow{\nein} g \notin G\\
			\downarrow \ja& \\
			\exists h_2 \in T_2 \colon a_2^{gh_1^{-1}h_2^{-1}} = a_2? & \xrightarrow{\nein} g \notin G\\
			\downarrow \ja & \\
			\dots & \\
			\exists h_r \in T_r \colon a_r^{gh_1^{-1}h_2^{-1}\cdots h_r^{-1}} = a_r? & \xrightarrow{\nein} g \notin G\\
			\downarrow \ja & \\
			\exists gh_1^{-1}h_2^{-1}\dots h_r^{-1} = e? & \xrightarrow{\nein} g \notin G\\
			\downarrow \ja & \\
			g \in G
		\end{matrix}
	\end{align*}
\end{conclusion}
\begin{proof}
	1. Schritt: Wegen $G = \bigcup_{h \in T_1} G_{a_1}h$ folgt\\
	\begin{align*}
		g \in G \equival \exists h \in T_1 \colon g \in G_{a_1}h \equival \exists h \in T_1 \colon gh^{-1} \in G_{a_1}\\
		\equival \exists h \in T_1 \colon a_1^{gh^{-1}} = a_1 \wedge gh^{-1} \in G \implies \exists h \in T_1 \colon a_1^{gh^{-1}} = a_1.
	\end{align*}
	Also führt der ($\xrightarrow{\nein}$)-Zweig zu $g \notin G$. (Die weiteren Schritte sind analog.)
\end{proof}
Problem: Wie findet man das Repräsentatensystem $T_1, \dots, T_r$ für die Untergruppen $U_1, \dots, U_r$, falls Erzeugendensystem $U$ gegeben
\begin{align*}
	G = \gen{U} \quad \text{ vergleiche (P3) und \cref}
\end{align*}
\section{Automorphismen, invariante Relationen und die Sätze von Krasner}
algebraisch haben wir folgende Sachen
\begin{align*}
	\begin{matrix}
		\text{kombinatorische Strukturen } &\leftrightarrow& \text{ Gruppe der ``Symmetrien''}\\
		\text{Relationen gegeben }&\rightarrow& \text{ Automorphismen}\\
		\text{Invariante Relation } &\leftarrow& \text{ $G$ gegeben}
	\end{matrix}
\end{align*}
\begin{*erinnerung}[\cref{sec2:exm:action_c}]
	$g \in S_M$ induziert $\tilde{g} \in S_{M^n}$ durch
	\begin{align*}
		(a_1,\dots,a_n)^g := (a_1, \dots,a_n)^{\tilde{g}} := (a_1^g, \dots,a_n^g)
	\end{align*}
	Bezeichnung der $(\tilde{G},M^n)$ auch mit $(G,M^n)$ oder $(G^{[n]},M^n)$.\\
	\cref{sec2:exm:action_a} $\implies$ Wirkung auf $\powerset(M^n)$\\
	Bezeichnung $(G, \powerset(M^n))$ (für $G \leqq S_M$:
	\begin{align*}
		\Phi^g := \set{\ul{a}^{\tilde{g}} \mid \ul{a} \in \Phi} \quad \text{ vergleiche \cref{sec1:def:multiplication_permu} für } \Phi \subseteqq M^n
	\end{align*}
\end{*erinnerung}
\begin{definition}
	$g \in S_M, \Phi \subseteqq M^n$ $n$-stellige Relation (Elemente ($n$-Tupel) als Spalten einer ``Matrix''). $g$ \begriff{bewahrt} $\Phi$ (Bezeichnung $g \keep \Phi$), also $\Phi$ \begriff{invariant} für $g$, bzw \scyllanote{$g$ Automorphismus von $\Phi$}\\
	\begin{align*}
		:\equival \Phi^g \subseteqq \Phi \text{ bzw. $\Phi^g = \Phi$}
	\end{align*}
	d.h. $\forall a_1, \dots, a_n \in M\colon (a_1,\dots,a_n) \in \Phi \overset{\scyllanote{\equival}}{\implies} a_1^g, \dots a_n^g \in \Phi$\\
	($M$ endlich: $g \keep \Phi \equival g$ Automorphismus)
\end{definition}
\begin{notation}
	Bezeichne die Menge der endlich-stelligen Relation mit
	\begin{align*}
		R_M &:= \set{\Phi \mid \subseteq M^n \mid n=1,2,3,\dots}\\
		\Aut \Phi = \Aut_M\Phi := \set{g \in S_M \mid \Phi^g = \Phi}\\
		\intertext{Für $\QQ \subseteq R_M$:}
		\Aut \QQ &:= \bigcap_{\Phi \in \QQ} \Aut \Phi \quad \text{(Automorphismen von $\QQ$)}\\
		n-\Inv (G,M) &:= n-\Inv_M G\;(n-\Inv G)\\
		&=\set{\Phi \subseteq M^n \mid \forall g \in G \colon g \keep \Phi}
	\end{align*}
	$n$-stellige Invarianten von $G$:\\
	\begin{align*}
		\Inv_M G := \bigcup_{n=1}^{\infty} n-\Inv G \quad \text{(\emph{Invarianten} von $G$)}
		\intertext{Jede binäre Relation, also auch}
		\set{(g,\Phi) \mid \Phi^g = \Phi} \subseteqq S_M \times R_M
	\end{align*}
	induziert eine \begriff{Galoisverbindung} $(\phi, \psi)$.	
\end{notation}
\begin{definition}
	\label{sec4:def:galois}
	Durch $\Aut$ und $\Inv$ ist eine Galoisverbindung gegeben:
	\begin{align*}
		\phi \colon \Aut \begin{cases}
			\powerset(R_M) &\to \powerset(S_M)\\
			\QQ &\mapsto \Aut \QQ\\
		\end{cases}\\
		\psi \colon \Inv \begin{cases}
			\powerset(S_M) &\to \powerset(R_M)\\
			G &\mapsto \Inv G
		\end{cases}
	\end{align*}
	insbesondere gelten die folgenden Eigenschaften $(G,G' \subseteq S_M, \QQ, Q'\subseteq R_M)$:
	\begin{defenum}
		\item $G \subseteqq G' \implies \Inv G \supseteqq \Inv G'$ \label{sec4:def:galois_a}
		\item $G \subseteqq Q' \implies \Aut \QQ \supseteqq \Aut Q'$ \label{sec4:def:galois_b}
		\item $G \subseteqq \Aut \Inv G$ \label{sec4:def:galois_c}
		\item $U \subseteq \Inv \Aut U$ \scyllanote{What is $U$ here? :o} \label{sec4:def:galois_d}
		\item $\Aut \Inv \Aut \QQ = \Aut \QQ$ \label{sec4:def:galois_e}
		\item $\Inv \Aut \Inv G = \Inv G$ \label{sec4:def:galois_f}
		\item $G \mapsto \Aut \Inv G$ ist Hüllenoperator \scyllanote{What is a Hüllenoperator?} \label{sec4:def:galois_g}
		\item $\QQ \mapsto \Inv \Aut \QQ$ ist Hüllenoperator \label{sec4:def:galois_h}
		\item $G \subseteqq \Aut \QQ \equival \Inv G \supseteqq \QQ$ \label{sec4:def:galois_i}
		\item $\Aut$ und $\Inv$ sind Bijektionen auf den Galoishüllen
		\begin{align*}
			G = \Aut \Inv G \quad \QQ = \Inv \Aut \QQ
		\end{align*} \label{sec4:def:galois_j}
	\end{defenum}
\end{definition}
\begin{*remark}
	\begin{itemize}
		\item \cref{sec4:def:galois_a}-\cref{sec4:def:galois_d} definieren bereits die Galoisverbindung.
		\item \cref{sec4:def:galois_e} - \cref{sec4:def:galois_j} sind Folgerungen aus \cref{sec4:def:galois_a}-\cref{sec4:def:galois_d}
	\end{itemize}
\end{*remark}
\begin{*erinnerung}[Hüllenoperator]
	content... %TODO add a small definition of a Hüllenoperator here!
\end{*erinnerung}
\begin{definition}
	\label{sec4:def:Orbitdef}
	Eine Relation der Form\begin{align*}
		(a_1,\dots, a_n)^G = \set{(a_1, \dots, a_n)^g \mid g \in G}
	\end{align*}
	heißt \begriff{$n$-Bahn} ($n$-Orbit) von $G \leqq S_M$.\\
	Notation:
	\begin{align*}
		n-\Orb(G,M) &= \text{ Menge der $n$-Bahnen}\\
		&= \set{\ul{a}^G \mid \ul{a} \in M^n}
	\end{align*}
\end{definition}
\begin{*remark}
	\begin{enumerate}
		\item $\Phi \in n-\Orb(G,M) \equival \Phi \in 1-\Orb(G^{[n]},M^n)$
		\item $\Phi \in \Inv(G,M) \equival \Phi$ invariante Menge von $(G^{[n]},M^n)$ (vergleiche \cref{sec1:exm:Orbit})
	\end{enumerate}
\end{*remark}
\begin{proposition}
	Sei $G \le S_M$.
	\begin{propenum}
		\item Jede $n$-Bahn ist eine invariante Relation:
		\begin{align*}
			n-\Orb(G,M) \subseteqq n-\Inv(G,M)
		\end{align*}\label{sec4:prop:relation_a}
		\item Jede $n$-stellige invariante Relation ist (disjunkte) Vereinigung von $n$-Bahnen \label{sec4:prop:relation_b}
		\item $\abs{n-\Inv (G,M)} = 2^{\abs{n-\Orb(G,M)}}$ \label{sec4:prop:relation_c}
	\end{propenum}
\end{proposition}
\begin{proof}
	\begin{enumerate}
		\item $\ul{a}^{G \cdot G} = \ul{a^G}$ für beliebige $\ul{a} \in M^n$ (wobei $\ul{a}^G$ $n$-Bahn ist)
		\item folgt aus \cref{lem_1.11_4} (\cref{sec4:prop:relation_b} für $n$-Bahnen) und Bemerkung zu \cref{sec4:def:Orbitdef}
	\end{enumerate}
\end{proof}
Folgerung aus \cref{prop:lagrange}.
\begin{lemma}
	Für $\Phi \in n-\Orb(G,M)$ und $\ul{a} = (a_1, \dots, a_n) \in \Phi$, gilt:
	\begin{align*}
		\abs{\Phi} = [G\colon G_{a_1, \dots,a_n}] = \frac{\abs{G}}{\abs{(a_1,\dots,a_n)}}.
	\end{align*}
	($G_{a_1, \dots,a_n}$ ist Stabilisator und \scyllanote{$\QQ=G^{[n]}$, letzteres gilt nach \cref{sec2:exm:actions}})
\end{lemma}
\begin{proof}
	\begin{align*}
		\Phi &= (a_1, \dots, a_n)^G =: \ul{a}^{\colortilde{G}}\\
		\colortilde{G}_{\ul{a}} &= G_{a_1\dots a_n} \text{für Wirkung } (\colortilde{G},M^n)\\
		&\xRightarrow{\cref{prop:lagrange}} \abs{\colortilde{G}} = \abs{G} = \abs{\colortilde{G}_{\ul{a}}}\cdot \abs{\ul{a}^{\colortilde{G}}} 
	\end{align*}
\end{proof}
Galoisverbindung $\Aut-\Inv$ (vergleiche \cref{sec4:def:Galois})\\
\begin{itemize}
	\item Was sind die Galoishüllen? (d.h. $\Aut \QQ$ bzw. $\Inv G$?)
	\item Probleme:
	\begin{itemize}
		\item Welche (Permutations)Gruppen sind Automorphismengruppen von geeigneten invarianten Relation?
		\item Welche Relationsmengen sind die invarianten Relationen für eine geeignete Gruppe $G \leqq S_M$?
	\end{itemize}
	\item Setze von \person{Maire Krasner} (1912-1985) (hier nur für endliche Grundmengen $M$)
\end{itemize}
Vorbemerkung:
\begin{proposition}
	\label{sec4:prop:Q_permu}
	Sei $\QQ \subseteq R_M$. Dann ist $\Aut_M \QQ$ eine (Permutations)Gruppe ($\le S_M$).
\end{proposition}
\begin{proof}
	SeSt!
\end{proof}
\begin{theorem}[1. Satz von \person{Krasner}]
	Sei $M = \set{a_1, \dots,a_m}$ endlich!
	\begin{theoenum}
		\item Jede Permutationsgruppe $(G,M)$ ist Automorphismengruppe einer geeigneten Menge von Relationen. Insbesondere gilt:
		\begin{align*}
			G &= \Aut \Inv G\\
			  &= \Aut \Orb G \text{ $\Orb$ alle $n$-Bahn, }n \in \set{1,2,3,\dots}\\
			  &= \Aut m-\Orb G\\
			  &= \Aut \ul{a}G \quad (\ul{a}:= (a_1, \dots, a_m)) 
		\end{align*}
		(Es reicht eine einzige $m$-stellige Relation)\label{sec4:theo:krasner_a}
		\item Für beliebige Teilmenge $G \subseteqq S_M$ gilt:
		\begin{align*}
			\gen{G} = \Aut \Inv G
		\end{align*}
		($\gen{G}$ interne Beschreibung der von $G$ erzeugten Untergruppe, $\Aut \Inv G$ externe Beschreibung der von $G$ erzeugten Untergruppe (als Galoishülle))\label{sec4:theo:krasner_b}
	\end{theoenum}
\end{theorem}
\begin{definition}
	\begin{enumerate}
		\item zu \cref{sec4:theo:krasner_a} Wir zeigen zunächst 
		\begin{align*}
			\Aut \Phi \subseteqq G
		\end{align*}
	für die von $\ul{a} = (a_1, \dots, a_m)$ erzeugte $m$-Bahn $\Phi = aG$. Sei $f \in \Aut \Phi \implies \underbracket{(a_1, \dots, a_m)}_{\in \Phi}^f = \ul{a}^{\ul{G}}$, also $\exists g \in G \colon (a_1, \dots, a_m)^f = (a_1, \dots, a_m)^g \in \ul{a}^{\ul{G}}$, d.h. $f = g \in G$, also $\Aut \Phi \subseteq G$.\\
	Die angegebenen Gleichungen folgen nun unmittelbar:
	\begin{align*}
		G \overset{\cref{sec4:def:galois_c}}&{\subseteqq} \Aut \Inv G \overset{\cref{sec4:def:galois_b}}{\subseteqq} \Aut \Orb G\\
		& \Aut m-\Orb G \subseteqq \Aut \set{\Phi} \subseteqq G.
	\end{align*}
	\item zu \cref{sec4:theo:krasner_b}
	\begin{align*}
		&G \subseteqq \Aut \Inv G \quad \cref{sec4:def:galois_c}\\
		&\implies \gen{G} \subseteqq \gen{\Aut \Inv G} \overset{\cref{sec4:prop:Q_permu}}{=} \Aut \Inv G \subseteq \Aut \Inv \gen{G} \overset{\cref{sec4:theo:krasner_a}}{=} \gen{G}.
	\end{align*}
	\end{enumerate}
\end{definition}
\begin{remark}[Operationen auf Relationen]\label{sec4:rem:oper_rela}
	Jede Formel $\phi(M, \dots,R_q, a_1, \dots, x_n)$ des Prädikantenkalküls 1. Stufe $(\exists, \forall, \vee, \wedge, \neg,=)$ und Relationssymbole (Prädikate) $R_1, \dots, R_q$ ($R_i$ sind $i$-stellig, $i = 1, \dots, q$) und freie Variablen $x_1, \dots, x_n$ definiert eine $q$-stellige Operation
	\begin{align*}
		Fq: \powerset(M^{r_i}) \times \dots \powerset(M^{r_i}) \to \powerset(M^n)
	\end{align*}
	(genau er \begriff{logische Operation}), die $q$ vielen Relationen $\Phi_1 \subseteq M^{r_1}, \dots , \Phi_g\subseteq M^{r_q}$ eine $n$-stellige Relation $F_{\phi}(\Phi_1,\dots,\Phi_q)$ zuordnet:
	\begin{align*}
		F_{\phi}(\Phi_1, \dots, \Phi_q) := \set{(a_1, \dots, a_q) \in M^n \mid \models \phi(\Phi_1, \dots, \Phi_q, a_1, \dots, a_n)}
	\end{align*}
	(wobei $\models$ ``es gilt'' heisst.)
\end{remark}
\begin{example}[logische Operationen]
	\begin{expenum}
		\item $\phi(R_1, R_2, x,y) :\equiv \exists z \colon R_1(x,y) \vee R_2(z,y)$
		\begin{align*}
			F_{\phi}(\Phi_1,\Phi_2) = \set{(x,y) \in M^2 \mid \exists z \colon \Phi_1(x,z) \vee \Phi_2(z,y)} = \phi_1 \circ \Phi_2 \quad \text{ Relationenprodukt}
		\end{align*}
		\item \begin{align*}
			\phi_{1\scyllanote{2}}(R_1,R_2,x,y) &:\equiv R_1(x,y) \vee \scyllanote{\wedge} R_2(x,y)\\
			F_{\phi_1, \scyllanote{\phi_2}}(\Phi_1, \Phi_2) &= \Phi_1 \cap \scyllanote{\cup} \Phi_2
		\end{align*}
		\item $\phi(R_1, x_1, \dots, x_n) :\equiv \neg R_1(x_1, \dots, x_n)$
		\begin{align*}
			F_{\phi}(\Phi_1)&=\neg \Phi_1 \quad (=M^n \setminus \Phi \text{ Komplement})
		\end{align*}
		\item $\phi(x_1, \dots, x_4) \vee x_1 = x_2 \vee x_3 = x_4$ (keine Prädikate für $q=0$) $\implies$ konstante Operation, 
		\begin{align*}
			F_{\phi} = \set{(a_1,a_2,a_3, a_4) \in M^4 \mid a_1 = a_2 \vee a_3 = a_4 \subseteq M^4}
		\end{align*}
		\item $\phi(x_1, x_2) := x_1 = x_2 \implies$ Konstante
		\begin{align*}
			F_{\phi} = \set{(a_1, a_2) \in M^2 \mid a_2 = a_1} = \Delta_M \quad \text{Gleichheitsrelation}
		\end{align*}
		\item $\phi(R_1, x_1, \dots, x_{i-1}, x_{i+1}, \dots, x_n) := \exists x_i \colon R_1(x_1, \dots, x_i, \dots, x_n)$
		\begin{align*}
			F_{\phi}(\Phi_1) &= \set{(a_1, \dots, a_{i-1}, a_{i+1},\dots, a_n) \in M^{n-1} \mid \exists a_i \colon (a_1, \dots, a_{i-1}, a_{i+1}, \dots, a_n) \in \Phi_1}\\
			&=: \pi_{n \setminus \set{i}}(\Phi_1) \text{ Projektion von $\Phi_1$ auf die $(\ul{n}\setminus \set{i})$-ten Koordinaten.}
		\end{align*}
		Streichen der $i$-ten Zeile (Bei Darstellung von Relationen durch ``Matrix'', Elemente ($n$-Tupel) als Spalten)
	\end{expenum}
\end{example}
\begin{definition}[Krasner-Algebra]
	Für $\QQ \subseteq R_M$ sei
	\begin{align*}
		[\QQ] := \set{F_{\phi}(\Phi_1, \dots, \Phi_q)\mid \Phi_1, \dots, \Phi_q \in \QQ, \phi(R_1, \dots, R_q, x_1, \dots, x_n)}
	\end{align*}
	formal wie in \cref{sec4:rem:oper_rela} mit $q \in \set{0,1,2, \dots}$, $n \in \set{1,2,\dots}$
\end{definition}
\subsection*{Abschluss gegen logische Operationen}
\begin{*remark}
	\begin{enumbem}
		\item $\QQ \mapsto [\QQ]$ ist Hüllenoperator (insbesondere gilt $[[\QQ]] = [\Q]$)
		\item Die abgeschlossene Mengen $\QQ$ (d.h. $\QQ = [\QQ]$) heißen auch \begriff{Krasner-Algebren}.
	\end{enumbem}
	Aus algebraischer Sicht sind dies genau die Unteralgebren von $\gen{R_M, (F_{\phi}_{\phi \text{ Formel}})}$. Äquivalente Beschreibung von Unteralgebren von
	\begin{align*}
		\ul{R_M} = \gen{R_M, \Delta_M, \neg, \rho, \tau, \Delta, \nabla, \circ}
	\end{align*}
	Dabei bedeuten die Elemente von $\ul{R_M}$ folgendes:
	\begin{itemize}
		\item $\Delta_M$: 
	\end{itemize}
\end{*remark}

\part*{Anhang}
\addcontentsline{toc}{part}{Anhang}
\appendix

\nocite{*}
%\bibliography{literatur}
%\bibliographystyle{acm}

%\printglossary[type=\acronymtype]

\printindex

\end{document}