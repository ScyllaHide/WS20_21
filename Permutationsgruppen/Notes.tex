% !TeX spellcheck = en_US
\documentclass[ngerman,a4paper,order=firstname]{mathscript}
\usepackage{mathoperators}
\usepackage{chemfig}

% % % local commands
\DeclareMathOperator{\Ad}{Ad}				% Adjoint
\DeclareMathOperator{\PSL}{PSL} 			% projective linear group 
\newcommand{\with}{\text{ with }}
\newcommand{\nd}{\text{ and }}
\newcommand{\for}{\text{ for }}
\renewcommand{\rhd}{\triangleright}
\renewcommand{\lhd}{\triangleleft} 			% normal subgroups
\DeclareMathOperator{\Set}{Set}				% Category of sets
\newcommand{\cat}{\mathcal C}				% some category
\DeclareMathOperator{\Vect}{Vect}			% Category of vector spaces
\DeclareMathOperator{\Grp}{Grp}				% Category of groups
\DeclareMathOperator{\QVect}{QVect}			% Cat of quasi vector bundles
\DeclareMathOperator{\Mod}{Mod}				% Cat of moduls
\newcommand{\EE}{\mathscr E}				% some cat E
\DeclareMathOperator{\Ann}{Ann}				% annihilator
\DeclareMathOperator{\morph}{Morph}				% cat of morphs
\newcommand{\Circlearrowleft}{\mathbin{\rotatebox[origin=c]{180}{$\circlearrowright$}}}
\DeclareMathOperator{\Cl}{Cl}				% conjugation class of something.
\renewcommand{\phi}{\varphi}				% always varphi for phi
\DeclareMathOperator{\Iso}{Iso}				% set of isomorphisms
\DeclareMathOperator{\Lin}{\mathscr L}		% set of continuous linear maps from V to W
\newcommand{\hset}[3]{\mathrm{H}^{#1}(#2;#3)} % set of equivalent cocycles over X
\newcommand{\isoset}[3]{\Phi^{#1}_{#2}(#3)} % isomorphism class of - F vector bundles
\DeclareMathOperator{\Open}{Open}				% inclusion category for presheafs
\newcommand{\presheaf}{\mathcal S}			% presheaf

% % % % color note stuff
%\newcommand{\marganote}[1]{\textcolor{gray}{#1}}
%\newcommand{\fehmnote}[1]{\textcolor{red}{#1}}
\newcommand{\scyllanote}[1]{\textcolor{blue}{#1}}

% Nice looking emptyset
\let\oldemptyset\emptyset
\let\emptyset\varnothing

% remove matrix columns limit
\setcounter{MaxMatrixCols}{20}

% get this stupid arrows:
%\usepackage{mathabx,graphicx}  % ---> add to mathoperators
%\def\Circlearrowleft{\ensuremath{%
%		\rotatebox[origin=c]{180}{$\circlearrowleft$}}}
%\def\Circlearrowright{\ensuremath{%
%		\rotatebox[origin=c]{180}{$\circlearrowright$}}}
%\def\CircleArrowleft{\ensuremath{%
%		\reflectbox{\rotatebox[origin=c]{180}{$\circlearrowleft$}}}}
%\def\CircleArrowright{\ensuremath{%
%		\reflectbox{\rotatebox[origin=c]{180}{$\circlearrowright$}}}}
%\begin{document}
%	\Huge
%	$\circlearrowleft \circlearrowright $
%	
%	$\Circlearrowleft \Circlearrowright $
%	
%	$\CircleArrowleft \CircleArrowright $

% % % local packages
\usepackage{braids}

\newlist{remarkenum}{enumerate}{1}
\setlist[remarkenum]{label=(\alph*),ref=\theremark~(\alph*)}
\crefalias{remarkenumi}{remark}

\newlist{propenum}{enumerate}{1}
\setlist[propenum]{label=(\alph*),ref=\theproposition~(\alph*)}
\crefalias{propenumi}{proposition}

\newlist{expenum}{enumerate}{1}
\setlist[expenum]{label=(\alph*),ref=\theexample~(\alph*)}
\crefalias{expenumi}{example}

\newlist{lemmaenum}{enumerate}{1}
\setlist[lemmaenum]{label=(\alph*),ref=\thelemma~(\alph*)}
\crefalias{lemmaenumi}{lemma}

\newlist{defenum}{enumerate}{1}
\setlist[defenum]{label=(\roman*),ref=\thedefinition~(\roman*)}
\crefalias{defenumi}{definition}

\newlist{theoenum}{enumerate}{1}
\setlist[theoenum]{label=(\roman*),ref=\thedefinition~(\roman*)}
\crefalias{theoenumi}{theorem}

\title{\textbf{Permutationsgruppen WS 20/21}}
\author{Prof. Pöschel}

\begin{document}
\pagenumbering{roman}
\pagestyle{plain}

\maketitle

\hypertarget{tocpage}{}
\tableofcontents
\bookmark[dest=tocpage,level=1]{Inhaltsverzeichnis}

\pagebreak
\pagenumbering{arabic}
\pagestyle{fancy}

\chapter*{Preface}
The plan is to go rather fast through the first chapter of this book(to get fast to K-Theory), take some notes, write down ideas, examples and remarks, rewrite proofs, so that i can understand them in my way. Also add sometimes reminders to concepts/definitions, so that i have a good overview about vector bundles and of course K-Theory. I will also use notation from courses i took in the past. But I will put remarks for the reader.
Hope you will find these notes helpful in any way.

ScyllaHide, \today
\chapter{Vector Bundles}
% !TeX spellcheck = en_US
\section{Topological Manifolds}
Lets recap a few definitions first, then we can define what a topological manifold is:
\begin{erinnerung}
	\begin{enumerate}
		\item topology on a set $M$ is a collection $\tau$ of  of subsets of $M$ we have the following
		\begin{enumerate}
			\item $\emptyset, M \in \tau$
			\item $\bigcup_{i=1}^{\infty} O_i \in \tau$ (infinite)
			\item $\bigcap_{i=1}^n O_i \in tau$
		\end{enumerate}
		then the pair $(M,\tau)$ is called topological space.
		\begin{itemize}
			\item Then we can define a subset $U \subset M$ is an open set if it belongs to the topology $\tau$.
			\item A neighborhood $U \in \tau$ of a point $p \in M$, contains simply $p \in M$.
			\item Closed set $F \subset M$, where the complement $M \setminus F = F^C$ is open.
			\item interior $\int A$ largest open set contained in $A$
			\item closure $\bar{A}$ smallest closed set containing $A$
			\item subset topology on $A\subset M$ is $\tau_A := (U \cap A)_{U \in \tau}$
		\end{itemize}
	\item quotient topology: quotient space $M\setminus sim$, where $\sim$ is an equivalence relation.
	define canonical projection $\pi\colon M \to M\setminus \sim$, the quotient topo is then, be letting $V \subset M\setminus \sim$ be open iff $\pi^{-1}(V) \subset M$ is open in $M$.
	\item product topology: 
	\end{enumerate}
\end{erinnerung}
\begin{definition}[topological manifold]
	M of dimension $n$ is a topological space with properties
	\begin{defenum}
		\item $M$ is Hausdorff for all $p_1\neq p_2 \in M$ there exists nbhs $V_1(p_1), V_2(p_2)$ st. $V_1 \cap V_2 = \emptyset$.\label{chap1_1:def:topomf:haus}
		\item $\forall p \in M,\phi\colon V(p) \subset M \to U_O \subset \R^n$ ($U_O$ open subset).\label{chap1_2:def:topomf:nbh}
		\item $M$ has second countability axiom, this means $M$ has countable basis for its topology.\label{chap1_2:def:topomf:basis}
	\end{defenum}
\end{definition}
\begin{example}[topological manifolds]
	\begin{expenum}
	\item $M \ubset \R^n$, then $(M,\tau_U)$ topological space, $U \subset M$ open iff $M\cap V = U$, $V_O \subset \R^n$.
	\item circle has dimension $n=1$
	\begin{align*}
		S^1 = \set{(x,y) \in \R^2 \mid x^2 + y^2 = 1},
	\end{align*}
	\cref{chap1_1:def:topomf:haus}, \cref{chap1_2:def:topomf:basis} inherited from ambient space $\R^2$. For \cref{chap1_2:def:topomf:nbh} we get a homeomorphism $\phi \colon U(p) \subset S^1 \to T \subsetset \R$, here $T$ open interval, choose the nbh $U(p)$ in such a way that the normal $n_p$ on this point is not parallel to a coordinate axis, then this projection is the needed homeomorphism. We basically bend (homeo) the nbh $U(p)$ on $S^1$, such that we have an open interval in $\R$. \scyllanote{Why we need the normal-part?}
	This can be generalized to $S^2$ (dimension $n=2$).
	\item surface of cube is topo mf its homeomorphic to $S^2$
	\end{expenum}
\end{example}
The next few examples use identification of a square.
\begin{example}
	\begin{expenum}
		\item Torus $\TT^2$ quotient of unit square $Q=[0,1]^2 \subset \R^2$ with equivalence relation(ER)
		\begin{align*}
			(x,y) \sim (x+1,y) \sim (x,y+1)
		\end{align*}
		with quotient topology.
		%TODO explain this a lil better ...
		\item\person{Klein} bottle $K^2$ quotient of $Q$ with ER
		\begin{align*}
			(x,y) \sim (x+1,y) \sim (1-x,y+1)
		\end{align*}
		\item projective place $\R P^2$ quotient of $Q$ by ER
		\begin{align*}
			(x,y)\sim (x+1,1-y) \sim (1-x,y+1)
		\end{align*}
	\end{expenum}
\end{example}
\begin{*example}
	the \person{Möbius}-strip is a topo mf with boundary.
\end{*example}
\begin{definition}[closed half space, topo mf with boundary, interior pts, bounary pts]
	%TODO here is not clear what \partial\HH^n?
\end{definition}
\section{Differential Manifolds}
%TODO
\section{Differentiable Maps}
%TODO
\section{Tangent Space}
%TODO
\section{Immersions and Embeddings}
%TODO
\section{Vectors Fields}
%TODO
An example to understand the interaction between vector fields, 1-parameter diffeomorphism group and local flow.
\begin{*example}
	Haben wir ein Vektorfeld $X$ auf $M$, für dass global ein Fluss $F:M \times (-\epsilon,\epsilon)\to M$ existiert, so gilt natürlich $F(x,0)=x$, also $F(0,\bullet)=\id_M$ ist ein Diffeomorphismus von $M$. Daher gilt auch auf einer kleinen Umgebung $U$ von 0, dass $F(x,u)$ für $u$ in $U$ ein Diffeomorphismus auf $M$ ist. Dann sieht man aber leicht ein, dass jedes $F(t,\bullet)$ als Komposition endlich vieler Faktoren $F(u,\bullet)$ geschrieben werden (mit $u$ in $U$). Daher ist $F(t,\bullet)$ für alle $t$ definiert und für alle $t$ ein Diffeomorphismus auf $M$. Wir erhalten also einen Homomorphismus $\R\to\Diffeo(M)$.\\
	Die Diffeomorphismengruppe $Diff(M)$ einer Mannigfaltigkeit ist einfach die Gruppe aller Diffeomorphismen $f\colon M\to M$ mit Komposition. $Diff(M)$ wirkt auf $M$.\\
	Ein einfaches Beispiel einer 1-Parameter-Untergruppe einer Diffeomorphismengruppe ist das folgende: Sei $M=\R$ und $X=1$ das Vektorfeld, das konstant 1 ist. Dann ist der zugehörige Fluss zu $X$: $F(x,t)=x+t$. Man sieht leicht, dass $F(F(x,t),s)=F(x,t+s)$.
\end{*example}
\section{Lie Groups}
%TODO
\section{Orientability}
%TODO
\section{Manifolds with Boundary}
%TODO

% !TeX spellcheck = en_US
\section{Topological spaces}
Collect stuff for the set-theoretic topology, to describe manifolds.
\begin{definition}[Topology]
	\proplbl{def_topology}
	$X$ is set and we define a collection $\tau$ of (open) subsets $U \subset X$. Then the pair $(X,\tau)$ is called topological space, if the following properties hold
	\begin{itemize}
		\item[(T1)] $X,\emptyset \in X$
		\item[(T2)] $\bigcup_{i \in I} U_i \in \tau$ ($I$ is index set and $\abs{I} \le \infty$)
		\item[(T3)] $\bigcap_{\stackrel{i \in J}{J\subsetneq I}}U_i \in \tau$ (finite, such that $\abs{J} < \infty$)
	\end{itemize}
\end{definition}
From now on $X$ abbreviation for topological space $(X,\tau)$.
\begin{*remark}
	\begin{itemize}
		\item Sometimes it is convenient \propref{def_topology} also to formulate with closed sets! (This is left for the reader.)
		\item easy examples of topo spaces are $(X,d)$ are metric spaces with $X = \Rn, \C^n$, etc (norms are equivalent on finite dimensional spaces. See ANAG 2017/2018)
	\end{itemize}
\end{*remark}
\begin{definition}[HAUSDORFF space, compact space]
	\proplbl{def_hausdorff} %TODO find out how to ref enumeration points :D
	\begin{defenum}
		\item $X$ is \begriff{hausdorff} if every open pair of points can separated by disjoint open neighborhoods, short
		\begin{align*}
		\forall x,y \in X, x\neq y \colon \exists U,V \open\colon U \cap V \neq \emptyset
		\end{align*}
		\item If $X$ is hausdorff, then we can define compactness, such that if every open cover has a finite subcover, such that
		\begin{align*}
			\bigcup_{i \in I} U_i \subseteq X, U_i \open \implies \exists i_i,\dots, i_n \in I \with \bigcup_{k=1}^n U_{i_k} \subseteq X
		\end{align*} 
	\end{defenum}
\end{definition}
\begin{definition}[topological base]
	\proplbl{def_topo_base}
	A system $\BB \subseteq \tau$ of open sets has a topological base, if every $U \in \tau$ can be realized as union of $U = \bigcup_{i \in I} B_i \with B_i \in \BB$.
\end{definition}
\begin{*remark}
	Well iv have seen here three axioms, have to check if these are already covered. Okay these are not axioms, it is how u can construct aa $\tau$ from a base u already have.
\end{*remark}
\begin{definition}[Second countable axiom]
	\proplbl{def_2nd_countable}
	$X$ is second countable or has the second countable property if the base $\BB$ is a countable (topoogical) base.
\end{definition}
\begin{example}
	Take the euclidean space $(\Rn, \tau)$ with its neighborhoods, then $\BB = \set{B_r(x) \mid x\in \Q^n, r \in \Q_{>0}}$ is a base and $(\Rn, \tau)$ is second countable.
\end{example}
\begin{definition}[$n$-dimensional topological manifold]
	Let $n \in \N$ non negative, $X$ has properties % dimension?! see karoubi? ohh well we look at \R^n stuff :o well that was the dim-definition for vb
	\propref{def_2nd_countable} and  \ref{def_hausdorff}, then we have an additional property and call it \begriff{$n$-topological manifold} $M$, that every $p \in M$ ($U \subseteq M$ open) is homeomorphic to $\Rn$.
\end{definition}
\begin{*remark}
	\begin{itemize}
		\item What means homeomorphic? $\exists \chi \colon U \to \Rn$ continuous, bijective, such that $\chi^{-1}\colon \Rn \to U$ also continuous.
		\item What happens if we replace $\Rn$ with $\C^n$? See \cite[p.~39]{Harder} for that.
		\item equivalent formulation for something?
		\item emphasize the definition of dimension, see remark above.
	\end{itemize}
\end{*remark}
\begin{definition}[differentiable atlas]
	Let $M$ be topological manifold with dimension $n$. A differentiable atlas $A$ on $M$ is an indexed collection ($(U,\chi)$ is a chart)
	\begin{align*}
		\AA = \set{(U_{\alpha},\chi_{\alpha})_{\alpha \in I} \mid U \subseteq M \open, \chi\colon U \overset{\cong}{\to} \Rn \text{ homeomorphism}}
	\end{align*}
	with the following properties
	\begin{enumerate}
		\item $M$ is covered by $U_{\alpha}$, such that $M = \bigcup_{\alpha \in I} U_{\alpha}$
		\item transition map between two charts is smooth: 
		\begin{align*}
			(U_{\alpha}, \chi_{\alpha}), (U_{\beta},\chi_{\beta}) \in \AA, U_{\alpha}\cap U_{\beta} \neq \emptyset
		\end{align*}
		and this implies 
		\begin{align*}
			\chi_{\alpha, \beta} = \chi_{\beta} \circ \chi_{\alpha} \colon \chi_{\alpha}(U_{\alpha}^{-1} \cap U_{\beta}) \to \chi_{\beta}(U_{\alpha} \cap U_{\beta})
		\end{align*}
		is smooth.
	\end{enumerate}
\end{definition}
% maybe create a figure here, eg https://en.wikipedia.org/wiki/Atlas_(topology)#/media/File:Two_coordinate_charts_on_a_manifold.svg
\begin{definition}[equivalent, compatible atlantes]
	Two atlantes $\AA, \AA'$ are \begriff{equivalent}, \begriff{compatible}, if $\AA \cup \AA'$ is also an atlas. This means
	\begin{align*}
		\forall (U_{\alpha},\chi_{\alpha}) \in \AA, (U_{\beta},\chi_{\beta}) \in \AA', U_{\alpha} \cap U_{\beta} \neq \emptyset\colon \chi_{\beta} \circ \chi_{\alpha}^{-1}\colon \chi_{\alpha}(U_{\alpha}\cap U_{\beta}) \to \chi_{\beta}(U_{\alpha} \cap U_{\beta})
	\end{align*}
	is smooth.
\end{definition}
\begin{example}
	\begin{itemize}
		\item Well we could take a look at the $\SS^1$ here and create two atlantes $\AA\nd \AA'$ and then think about when $\AA \sim \AA'$?
		\begin{itemize}
			\item representation of $\SS^1$ in polar coordinates and then define two ways to cover $\SS^1$.
			\item Of course we have $\AA \cup \AA'$ is again an atlas, maybe ``throw some charts away''
			\item we need to find those $\chi_{\beta} \circ \chi_{\alpha}^{-1}$ and prove these are smooth again. % well \chi_i are already there, might need a drawing for that.
			
		\end{itemize}
		\item here is a not example, when the $\chi_{\beta} \circ \chi_{\alpha}^{-1}$ are not smooth? 
		Okay lets write these two atlantes down.
		\begin{align*}
			\AA &:= \set{(\R, \id \colon \R \to \R)}\\
			\AA' &:= \set{(\R, \sqrt[3]{\cdot}\colon \R \to \R)}
		\end{align*}
		the culprit is probably $\sqrt[3]{\cdot}$ is not smooth and therefore $\sqrt[3]{\id^{-1}}$ is not smooth. % maybe here? https://math.stackexchange.com/questions/1167558/cube-root-function-not-differentiable
		\item is this fixable by remove points or restrict?
	\end{itemize}
\end{example}
\section{Smooth Manifolds}
\begin{definition}
	A smooth (= differentiable) manifold $M$ is a topological space together with an equivalence class of atlantes (a ``smooth structure''). We can write this as $(M,~)$.
\end{definition}
From now on, we assume all manifolds have the property to be smooth!
\begin{example}
	Here is a bunch of manifolds ... %TODO type them.
	\begin{enumerate}
		\item 
		\item 
		\item 
		\item 
		\item 
		\item 
	\end{enumerate}
\end{example}
\section{Smooth Maps}
\begin{definition}
	Let $N,M$ two manifolds. A map $f\colon M \to N$ is smooth, if for every pair $(U,\chi) \nd (V,\psi)$ on $M$, $N$ holds: $\psi \circ f \circ \chi$ smooth
	\[
		% https://tikzcd.yichuanshen.de/#N4Igdg9gJgpgziAXAbVABwnAlgFyxMJZABgBpiBdUkANwEMAbAVxiRAFUACAHW7ibRwYOGAEdOAWRABfUuky58hFACZyVWoxZsAcjz5MARkJHiAajLkgM2PASJqAjBvrNWiELwBKAPQC2lvK2SkRkztSu2h7ePoTSGjBQAObwRKAAZgBOEAGIZCA4EEiOEVruIOmBFdm5JQVFiGqabmy8glhVWTlI+YVIAMylLdHcAMYAFh2yGTUD1H2NQ1Ge3O36o1iZo5zp65vbvBNYPsAAtI7SMhTSQA
		\begin{tikzcd}
			U \supseteq M \arrow[rr, "f"] \arrow[d, "\chi"] &  & N \subseteq V \arrow[d, "\psi"] \\
			\R^n \arrow[rr, "\psi \circ f \circ \chi^{-1}"] &  & \R^m                           
		\end{tikzcd}	
	\]
\end{definition}
\begin{*remark}
	\begin{itemize}
		\item This means a map is smooth, if the map is smooth in local coordinates.
		\item notation: $c^{\infty}(M,N) := \set{f\colon M \to N \text{ smooth}}$, $C^{\infty} := C^{\infty}(M, \R)$
		\item composition of smooth maps is smooth again (check)
		\item $C^{\infty}$ is $\R$-Algebra, sums, multiples of smooth maps is smooth again (check)
	\end{itemize}
\end{*remark}
\begin{definition}
	A smooth map $f\colon M \to N$ is called \begriff{diffeomorphism} if $\exists g \colon N \to M$ smooth
	\begin{align*}
		g \circ f = \id_M \nd f \circ g = \id_N
	\end{align*}
	If exists a diffeomorphism $f \colon M \to N$, then we call $M,N$ diffeomorphic, which note by
	\begin{align*}
		M \cong N, M \xrightarrow[f]{\cong} N
	\end{align*}
\end{definition}
\begin{*remark}
	is be diffeomeorphic an equivalence relation?
\end{*remark}
then we list a few simple examples to illustrate
\begin{example}
	\begin{itemize}
		\item $(\R, \AA_1 = \set{(\R, \id)}) \cong (\R, \AA_2 = \set{(\R, \sqrt[3]{\cdot})})$ compare that $\AA_1 \nd \AA_2$ are not equivalent atlantes.
		\item $B_1(0) \cong \R^n$ with $x \mapsto \tan(\pi/2 \norm{x})\cdot x$
		\item two atlantes $\AA_1,\AA_2$ on $M$ are equivalent, if
		\begin{align*}
			\id \colon (M,\AA_1) \to (M,\AA_2)
		\end{align*}
		is a diffeomorphism and $\id^{-1}$ too.
	\end{itemize}
\end{example}
\section{tangent space and tangent bundle}
\begin{definition}[tangent space]
	Let $p \in M$. \begriff{tangent space} of $M$ at $p$ is defined as the space of derivations of $C^{\infty}(M)$ on $p$
	\begin{align*}
		T_p M := \set{ \der\colon C^{\infty}(M) \to \R \mid \der \text{ linear, }\der(fg) = f(p)\der(g) + g(p)\der(f)}
	\end{align*}
\end{definition}
\begin{proposition}
	Inclusion map $i\colon Z \hookrightarrow M$ induces an isomorphism $\pull{\pull{i}} \colon T_p U \to T_p M \with \der \circ \pull{i}$. Here is $\pull{i} \colon C^{\infty}(M) \to C^{\infty}(U)$ the pullback map, see \propref{def_pullback_map}.
\end{proposition}
\begin{proof}
	Analog to proof of \propref{prop_isobtwTS}, construct cutoff function in $\chi(U) \subseteq \R^n$ and transport on $U$.
\end{proof}
\begin{conclusion}
	If $\dim M = n$, then $T_p M \cong \R^n$ with $p \in M$.
\end{conclusion}
\subsection{Differential of a map}


\part*{Anhang}
\addcontentsline{toc}{part}{Anhang}
\appendix

\nocite{*}
%\bibliography{literatur}
%\bibliographystyle{acm}

%\printglossary[type=\acronymtype]

\printindex

\end{document}